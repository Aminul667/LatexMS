\documentclass[11pt,a4paper]{article}
\usepackage[margin=1in]{geometry}
\usepackage{amsfonts,amsmath,amssymb,suetterl}
\usepackage{lmodern}
\usepackage[T1]{fontenc}
\usepackage{fancyhdr}
\usepackage{float}
\usepackage[utf8]{inputenc}
\usepackage{physics}
\usepackage{fontawesome}
\DeclareUnicodeCharacter{2212}{-}
\usepackage{mathrsfs}
\usepackage[nodisplayskipstretch]{setspace}

\setstretch{1.5}
\setlength{\headheight}{15.2pt}

\pagestyle{fancy}
\fancyfoot[C]{\thepage}
\fancyfoot[L]{Course Organizer: R. J. HARRIS}
\fancyfoot[R]{rosemary.harris@qmul.ac.uk}

% \renewcommand{\footrulewidth}{0pt}
\renewcommand{\headrulewidth}{0pt}
\parindent 0ex
\setlength{\parskip}{1em}
\allowdisplaybreaks

\begin{document}
  \textbf{MTH734U/MTHM012} \hfill \textbf{Mathematical Sciences}\\
  \textbf{Semester B 2010 - 2011} \hfill \textbf{QMUL}
  \begin{center}
    \textbf{\huge Week 10}
  \end{center}
  \hrule \vspace{2mm} \hrule

  \section*{Key Objective:}
  \textit{Know the definition of Brownian Motion. Be able to calculate the means and covariance functions of related stochastic processes and determine whether, or not, they are standard Brownian motions}

  \section*{Background:}
  Please read the whole of [T+K] Chapter VIll, Section 1 which expands on the topics covered in the lecture. To provide a solid foundation for future work, I recommend trying to do the (easier) "Exercises" at the end of the chapter as well as the "Problems" listed overleaf.
  \newpage

  \section*{Problems:}
  \begin{enumerate}
    \item \textbf{[Part question from 2007 exam paper]}\\
    Let $B(t)$ be standard Brownian motion. Show that $tB(1/t)$ is again standard Brownian motion.
    \item \textbf{Product moments}\\
    Consider a standard Brownian motion $\{B(t): t \geq 0\}$ at times $0 < t_l < t_2 < t_3 < t_4$.
    \begin{enumerate}
      \item Evaluate the product moment
      $$
      E[B(t_1)B(t_2)B(t_3)].
      $$
      \item Evaluate the product moment
      $$
      E[B(t_1)B(t_2)B(t_3)B(t_4)].
      $$
    \end{enumerate}
    \item \textbf{[Expectation of a process related to Brownian motion]}\\
    Do [T + K] Problem VIII.1.2.
    \item \textbf{Reflected Brownian Motion}\\
    Do [T + K] Problem VIII.1.3 - also part of a question on the 2006 exam paper.
    \item \textbf{Random walk, invariance principle}\\
    Do [T + K] Problem V111.1.5 - quite long, consider it as an optional challenge.
  \end{enumerate}
  \textbf{(Hints available in tutorial on 23rd March, solutions in tutorial on 30th March)}
  \newpage

  \textbf{MTH734U/MTHM012} \hfill \textbf{Mathematical Sciences}\\
  \textbf{Semester B 2010 - 2011} \hfill \textbf{QMUL}
  \begin{center}
    \textbf{\huge Solutions for Week 10}
  \end{center}
  \hrule \vspace{2mm} \hrule

  \section*{General comments:}
  The first three problems are all pretty standard so make sure you understand the details. Similar questions for practice can be found in the past exam papers from 2005—2007.\\
  In what follows, unless explicitly stated otherwise, I use $B(t)$ to denote standard Brownian motion with $B(O)= 0$. I also use $\Phi(z)$ for the cumulative distribution function of the standard normal distribution and $\phi(z)$ for the corresponding p.d.f., i.e,
  $$
  \Phi(x) = \int_{-\infty}^x\phi(u)du = \int_{-\infty}^x\frac{1}{\sqrt{2\pi}}e^{-\frac{1}{2}u^2}du.
  $$
  Notice that
  \begin{align*}
    \Phi\left(\frac{x}{\sqrt{t}}\right)
    &= \int_{-\infty}^\frac{x}{\sqrt{t}}\phi(u)du\\
    &= \int_{-\infty}^x\frac{1}{\sqrt{t}}\phi\left(\frac{x}{\sqrt{t}}\right)dz.
  \end{align*}
  Note the factor of $\frac{1}{\sqrt{t}}$ which results from changing the variables. In particular, the transition density for standard Brownian motion should read
  $$
  p(y, t|x) = \frac{1}{\sqrt{t}}\phi\left(\frac{y - x}{\sqrt{t}}\right) = \frac{1}{\sqrt{2\pi t}}\exp\left(-\frac{1}{2t}(y - x)^2\right).
  $$

  \section*{Solutions to Problems:}
  \begin{enumerate}
    \item \textbf{[Part question from 2007 exam paper]}\\
    Let $\tilde{B}(t) = tB(1/t)$. Note that since $B(t)$ is a Gaussian process so $\tilde{B}(t)$ is also Gaussian. Also $B(t)$ is continuous so $\tilde{B}(t)$ is continuous. [You might (legitimately) worry about proving continuity at $t = 0$ but it seems the exam question intended you to ignore this technicality]. To demonstrate Brownian motion we now need merely to check the mean and covariance.
    \begin{itemize}
      \item Mean: $E[\tilde{B}(t)] = tE[B(1/t)] = 0$
      \item Covariance function:
      \begin{align*}
        \text{Cov}[\tilde{B}(s), \tilde{B}(t)]
        &= \text{Cov}[sB(1/s)tB(1/t)]\\
        &= st\text{Cov}[B(1/s)B(1/t)]\\
        &= st\min\left\{\frac{1}{s}, \frac{1}{t}\right\}\\
        &= \min \{s,t\}
      \end{align*}
      corresponding to $\sigma^2 = 1$.\\
      Arguably, to check that it's standard Brownian motion we also need to show $\tilde{B}(0) = 0$. Indeed, assuming continuity as above, $\tilde{B}(0) = \lim_{t\to\infty} \tilde{B}(t) = 0$.\\
      So we conclude, as required, that $\tilde{B}(t)$ is also standard Brownian motion.
    \end{itemize}
    \item \textbf{Product moments}
    \begin{enumerate}
      \item Repeated application of the independent increments property yields
      \begin{align*}
        E[B(t_1)B(t_2)B(t_3)] 
        &= E[B(t_1)B(t_2)\{B(t_3) - B(t_2) + B(t_2)\}]\\
        &= E[B(t_1)B(t_2)]E[B(t_3) - B(t_2)] + E[B(t_1)B(t_2)B(t_2)]\\
        &= t_1\times 0 + E[B(t_1)\{B(t_2) - B(t_1) + B(t_1)\}^2]\\
        &= E[B(t_1)]E[\{B(t_2) - B(t_1)\}^2] + 2E[B(t_1)^2]E[B(t_2) - B(t_1)] \\
        &  \qquad + E[B(t_1)^3]\\
        &= 0\times (t_2 - t_1) + 2\times t_1\times 0 + 0\\
        &= 0,
      \end{align*}
      as might have been obvious in the first place from symmetry considerations.
      \item Similarly,
      \begin{align*}
        E[B(t_1)B(t_2)B(t_3)]
        &= E[B(t_1)B(t_2)B(t_3)\{B(t_3) - B(t_2) + B(t_2)\}]\\
        &= E[B(t_1)B(t_2)B(t_3)]E[B(t_4) - B(t_3)] + E[B(t_1)B(t_2)B(t_3)^2]\\
        &= 0\times 0 + E[B(t_1)B(t_2)\{B(t_3) - B(t_2) + B(t_2)\}^2]\\
        &= E[B(t_1)B(t_2)]E[\{B(t_3) - B(t_2)\}^2] + 2E[B(t_1)B(t_2)^2]E[B(t_3) \\
        & \quad - B(t_2)] + E[B(t_1)B(t_2)^3]\\
        &= t_1(t_3 - t_2) + 0 \times 0 + E[B(t_1)\{B(t_2) - B(t_1) + B(t_1)\}^3]\\
        &= t_1(t_3 - t_2) + E[B(t_1)]E[\{B(t_2) - B(t_1)\}^3] + 3E[B(t_1)^2]\\
        & \qquad E[\{B(t_2) - B(t_1)\}^2] + 3E[B(t_1)^3]E[\{B(t_2) - B(t_1)\}] \\
        & \qquad + E[B(t_1)^4]\\
        &= t_1(t_3 - t_2) + 0 \times 0 + 3t_1(t_2 - t_1) + 3 \times 0 \times 0 + 3(t_1)^2\\
        & \qquad\qquad\qquad\qquad\qquad\qquad\qquad\text{[moments of Gaussian]}\\
        &= t_1(t_3 + 2t_2).
      \end{align*}
    \end{enumerate}
    \item \textbf{Expectation of a process related to Brownian motion}
    \begin{align*}
      E[e^{\lambda B(t)}]
      &= \int_{-\infty}^\infty e^{\lambda x}\frac{1}{\sqrt{2\pi t}}e^{-x^2/2t}dx\\
      &= e^{\lambda^2t/2}\int_{-\infty}^\infty\frac{1}{\sqrt{2\pi t}}e^{-(x - \lambda t)^2/2t}dx\quad\text{[completing square]}\\
      &= e^{\lambda^2t/2}\int_{-\infty}^\infty\frac{1}{\sqrt{2\pi t}}e^{-y^2/2t}dy\quad\text{[changing variable $y = x - \lambda t$]}\\
      &= e^{\lambda^2 t/2}.\quad\text{[Gaussian integral]}
    \end{align*}
    This is an example of Geometric Brownian motion which plays an important role in modelling stock market prices.
    \item \textbf{Reflected Brownian Motion}
    \begin{align*}
      \text{Pr}\left\{\frac{|B(t)|}{t} > \epsilon \right\}
      &= \text{pr}\left\{\frac{B(t)}{t}< - \epsilon\right\} + \text{Pr}\left\{\frac{B(t)}{t} > \epsilon\right\}\\
      &= 2\text{pr}\left\{\frac{B(t)}{t} > \epsilon\right\}\\
      &= 2[1 - \text{Pr}\{B(t) \leq \epsilon t\}]\\
      &= 2[1 - \Phi(\epsilon t/\sqrt{t})]\\
      &= 2[1 - \Phi(\epsilon\sqrt{t})].
    \end{align*}
    Remembering that
    $$
    \Phi(x) = \int_{-\infty}^x\frac{1}{\sqrt{2\pi}}e^{-\frac{1}{2}u^2}du,
    $$
    it's easy to extract the limiting behaviour:
    \begin{itemize}
      \item As $x \to \infty,\, \Phi(x)\to 1$ so 
      $$
      \text{Pr}\left\{\frac{B(t)}{t} > \epsilon\right\}\to 0 \quad\text{as $t \to \infty$}.
      $$
      [To see the form of this decay you can use the approximation $\Phi(x) \approx 1 - \phi(x)/x $ for $x$ large.]
      \item For $x\approx 0,\, \Phi(x) \approx 0.5$ so
      $$
      \text{Pr}\left\{\frac{|B(t)|}{t} > \epsilon\right\} \approx 1\quad \text{for $t \approx 0$}.
      $$
    \end{itemize}
    \item \textbf{Random walk, invariance principle}
    \begin{enumerate}
      \item First we note that the required probability is given by
      $$
      \text{Pr}\{M_\tau = 0\} = \text{Pr}\{\text{$S_n$ reaches $-a$ before 1 | $S_n = 0$}\}
      $$
      To calculate this, we use (as directed) first step analysis. This is standard but somewhat tedious; I outline the procedure below and refer you to Sec. 111.6 of [T+K] for more details. Let $u_i$ be the probability that $S_n$ reaches $-a$ before 1 given that $S_0 = i$, i.e,
      $$
      u_i = \text{Pr}\{\text{$S_n$ reaches $-a$ before 1 | $S_0 = i$\}}.
      $$
      (In this notation we ultimately want to calculate $u_0$.) Considering the position of the random walk after a single step from the starting point $i$, yields the difference equation
      \begin{equation}\tag{10.1}
        u_i = \frac{1}{2}u_{i + 1} + \frac{1}{2}u_{i - 1}\quad\text{for $i = 0, -1, -,\ldots,-(a - 1)$}
      \end{equation}
      with (obvious) boundary condition
      $$
      u_1 = 0,\quad\text{and}\quad u_{-a} = 1.
      $$
      The trick to solve these equations is to consider the difference $x_i = u_i - u_{i - 1}$ for $i = 1, 0, -1,\ldots,-(a - 1)$. Simple rearrangement of (10.1) gives
      $$
      0 = x_{i + 1} - x_i,\quad \text{for $i = 0, -1, -2,\ldots,-(a-1)$}
      $$
      and it follows that $x_i = c$ where $c$ is a constant. To determine $c$ we sum the $x_i$'s and use the boundary conditions:
      \begin{align*}
        \sum_{i = -(a - 1)}^1 x_i &= u_1 - u_{-a}\\
        (a + 1)c &= 0 - 1\\
        c &= -\frac{1}{1 + a}.
      \end{align*}
      Finally, the boundary condition $u_1 = 0$ implies
      $$
      x_1 = -u_0,
      $$
      so that the required probability is given by
      \begin{align*}
        \text{Pr}\{M_\tau = 0\}
        &= u_0\\
        &= -x_1\\
        &= \frac{1}{1 + a}.\quad\text{[cf. [T + K] (1.12) with $b = 1$]}
      \end{align*}
      \item Consider the random walk $S_n$ starting at $S_0 = 0$. If $M_\tau \geq 2$, then we must obviously also have $M_\tau \geq 1$. This implies that the random walk reaches $S_n = 1$ at some time, $m$ say, before it reaches $S_n = -a$. Since the process is Markovian what happens for times $n > m$ is independent of the history for $n < m$. Furthermore, the random walk probabilities are homogeneous (i.e., constant) in time and space. Thus we can consider the random walk to start afresh from $S_m = 1$ and, conditional on this event, the probability to reach $S_n = 2$ before dropping a units to $S_n = -(a - 1)$ must be the same as $\text{Pr}\{\text{$S_n$ reaches 1 before $-a$ | $S_0 = 0$}\} = \text{Pr}\{M_\tau \geq 1\}$. Hence, we have,
      \begin{align*}
        \text{Pr}\{M_\tau \geq 2\}
        &= \text{Pr}\{M_\tau \geq 1\}\times\text{Pr}\{M_\tau \geq 1\}\\
        &= \text{Pr}\{M_\tau \geq 1\}^2
      \end{align*}
      where
      \begin{align*}
        \text{Pr}\{M_\tau \geq 1\}
        &= 1 - \text{Pr}\{M_\tau = 0\}\\
        &= \frac{a}{1 + a}
      \end{align*}
      Continuing the argument (or formally proving by induction) we see that, for $k = 1, 2, 3,\ldots,$
      \begin{equation}\tag{10.2}
        \begin{aligned}[b]
          \text{Pr}\{M_\tau \geq k\}
          &= \text{Pr}\{M_\tau \geq k - 1\}\text{Pr}\{M_\tau \geq 1\}\\
          &= \text{Pr}\{M_\tau \geq 1\}^k\\
          &= \left(\frac{a}{1 + a}\right)^k.
        \end{aligned}
      \end{equation}
      We conclude that MT has the geometric distribution,
      $$
      \text{Pr}\{M_\tau = k\} = \left(\frac{1}{1 + a}\right)\left(\frac{a}{1 + a}\right)^k.
      $$
      \item First we construct approximate Brownian motion
      $$
      B_n(t) = \frac{S_{[nt]}}{\sqrt{n}}
      $$
      where $[x]$ is the greatest integer less than or equal to $x$. Notice that this rescales the lengths in the problem by a factor of $\sqrt{n}$. The invariance principle states that as $n \to \infty$, $B_n(t)$ converges to standard Brownian motion $B(t)$. Rescaling the lengths in (10.2) and taking the limit, we find
      \begin{align*}
        \text{Pr}\{M(\tau) \geq x\}
        &= \lim_{n\to \infty}\left(\frac{a\sqrt{n}}{1 + a\sqrt{n}}\right)^{x\sqrt{n}}\\
        &= \lim_{n\to \infty}\left(\frac{1}{1 + \frac{1}{a\sqrt{n}}}\right)^{x\sqrt{n}}\\
        &= \lim_{n\to\infty}\left(1 - \frac{1}{a\sqrt{n}}\right)^{x\sqrt{n}}\\
        &= e^{-x/a}.
      \end{align*}
      Hence
      $$
      \text{Pr}\{M(\tau) \leq x\} = 1 - e^{-x/a},
      $$
      i.e., $M(\tau)$ has an exponential distribution with mean $a$.
    \end{enumerate}
  \end{enumerate}
\end{document}