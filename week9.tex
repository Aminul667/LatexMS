\documentclass[11pt,a4paper]{article}
\usepackage[margin=1in]{geometry}
\usepackage{amsfonts,amsmath,amssymb,suetterl}
\usepackage{lmodern}
\usepackage[T1]{fontenc}
\usepackage{fancyhdr}
\usepackage{float}
\usepackage[utf8]{inputenc}
\usepackage{physics}
\usepackage{fontawesome}
\DeclareUnicodeCharacter{2212}{-}
\usepackage{mathrsfs}
\usepackage[nodisplayskipstretch]{setspace}

\setstretch{1.5}
\setlength{\headheight}{15.2pt}

\pagestyle{fancy}
\fancyfoot[C]{\thepage}
\fancyfoot[L]{Course Organizer: R. J. HARRIS}
\fancyfoot[R]{rosemary.harris@qmul.ac.uk}

% \renewcommand{\footrulewidth}{0pt}
\renewcommand{\headrulewidth}{0pt}
\parindent 0ex
\setlength{\parskip}{1em}

\begin{document}
  \textbf{MTH734U/MTHM012} \hfill \textbf{Mathematical Sciences}\\
  \textbf{Semester B 2010 - 2011} \hfill \textbf{QMUL}
  \begin{center}
    \textbf{\huge Week 9}
  \end{center}
  \hrule \vspace{2mm} \hrule
  \section*{Key Objective:}
  \textit{Know the ergodic theorem for continuous-time Markov chains and be able to find the stationary distributions for simple examples.}

  \section*{Background:}
  You should now be comfortable with all the material in [T+KJ Sec. VI.6. The ergodic theorem for continuous-time Markov chains is not really covered in the book so please make sure to be familiar with the formulation presented in the lecture notes and/or check out more advanced references such as Grimmett and Stirzaker. In terms of application, you need to be able to construct the generator for simple scenarios and then find the corresponding stationary distribution — see examples in [T+K] and the problems below.

  \section*{Problems:}
  \begin{enumerate}
    \item \textbf{On-off process again}\\
    Do [T + K] Problem VI.6.2. We did this problem a few weeks ago within the framework of renewal theory. Now treat each component as a Markov chain and compare your results. Which method do you find easier?
    \item \textbf{Student health}\\
    Sam Lacker is an undergraduate student whose health appears to be a Markov process fluctuating between three states: 0 (fit), 1 (minor illness which prevents him studying but does not stop him working at his part-time job), 2 (major illness requiring him to stay at home). When he develops an illness from the fit state it has a probability $\frac{1}{3}$ to be a minor illness and a probability $\frac{2}{3}$ to be a major illness. From a minor illness, he recovers with probability $\frac{1}{3}$ or develops a major illness with probability $\frac{2}{3}$. From a state Of major illness he always returns directly to full fitness. Assuming time units of days, the sojourn times in states $0, 1, 2$ are exponentially distributed with parameters $\frac{1}{3},\, \frac{1}{3},\, \frac{1}{6}$ respectively.
    \begin{enumerate}
      \item What is the generator matrix?
      \item What is the long-run proportion of time Sam is able to study?
      \item If Sam earns $\pounds 20$ per daily shift worked, what are his long-run weekly earnings?
    \end{enumerate}
    \item \textbf{Computer queue}\\
    Jobs arrive in a computer queue in the manner of a Poisson process with intensity $\lambda$. The central processor handles them one by one in the order of their arrival, and each has an exponentially distributed runtime with parameter $\mu$, the runtimes of different jobs being independent of each other and of the arrival process. Let $X(t)$ be the number of jobs in the system (either running or waiting) at time $t$, where $X(0) = 0$. Explain why $X$ is a Markov chain, and write down its generator. Show that a stationary distribution exists if and only if $\lambda < \mu$, and find it in this case.
  \end{enumerate}
  \textbf{(Hints available in tutorial on 16th March, solutions in tutorial on 23rd March)}

  \newpage
  \textbf{MTH734U/MTHM012} \hfill \textbf{Mathematical Sciences}\\
  \textbf{Semester B 2010 - 2011} \hfill \textbf{QMUL}
  \begin{center}
    \textbf{\huge Solutions for Week 9}
  \end{center}
  \hrule \vspace{2mm} \hrule

  \section*{General comments:}
  All of these questions rely on the use of the equation $\pi \vb{A} = 0$, to find the stationary distribution $\pi$ from knowledge of the generator $\vb{A}$

  \section*{Solutions to Problems:}
  \begin{enumerate}
    \item \textbf{On-off process again}\\
    Since each component has two states (0 or 1) the complete system has eight states and we could model it as an eight-state Markov chain. However, it is much simpler to consider each component separately as a two-state Markov chain and use their statistical independence. The generator for component $x$ (where $x\in \{A, B, C\}$) is given by
    $$
    \vb{A}_x =
    \begin{cases}
      -\alpha_x & \alpha_x\\
      \beta_x & -\beta_x
    \end{cases}.
    $$
    Let $\pi_x = (1 - p_x, p_x)$ be the stationary probability distribution corresponding to this generator. In other words, $p_x$ is the long-run probability that component is operating. For a stationary distribution we require $\pi_x\vb{A}_x = 0$ so
    $$
    (1 - p_x, p_x)
    \begin{pmatrix}
      -\alpha_x & \alpha_x\\
      \beta_x & -\beta_x
    \end{pmatrix}
    = 0
    $$
    This yields
    $$
    (1 - p_x)\alpha_x + p_x\beta_x = 0
    $$
    and hence
    $$
    p_x = \frac{\alpha_x}{\alpha_x + \beta_x}.
    $$
    Remembering that the components are assumed independent, the probability that neither of B or C is operating is $(1 - \beta_B)(1 - \beta_C)$; hence the probability that at least one of them is working is $1 - (1 - p_B)(1 - p_C)$. It follows that the probability $p$ that the complete system is operating is given by
    \begin{align*}
      p
      &= p_A[1 - (1 - p_B)(1 - p_C)]\\
      &= p_A[p_B + p_C - p_Bp_C]\\
      &= \frac{\alpha_A[\alpha_B(\alpha_C + \beta_C) + (\alpha_B + \beta_B)\alpha_C - \alpha_B\alpha_C]}{(\alpha_A + \beta_A)(\alpha_B + \beta_B)(\alpha_C + \beta_C)}\\
      &= \frac{\alpha_A(\alpha_B\alpha_C + \alpha_B\beta_C + \beta_B\alpha_C)}{(\alpha_A + \beta_A)(\alpha_B + \beta_B)(\alpha_C + \beta_C)}
    \end{align*}
    [It's largely a matter of personal taste, whether you prefer to use the renewal process formalism or the Markov chain approach to find the long-run probabilities. Note, however, that the renewal process method is more general since, in principle, it can also deal with operating times which are not exponentially distributed.]
    \item \textbf{Student health}
    \begin{enumerate}
      \item First note that the diagonal elements Of the generator are given by $A_{ii} = -q_i$ where $q_i$ is the inverse of the mean sojourn time in state $i$. After leaving state $i$ the probability that the process jumps to state $j \neq i$ is given by $p_{ij} = q_{ij}/q_i$. The question effectively gives us the $p_{ij}$'s so we can calculate the $q_{ij}$'s which are the off-diagonal elements of the generator. The complete generator matrix is thus given by
      $$
      \vb{A} =
      \begin{pmatrix}
        -\frac{1}{3} & \frac{1}{9} & \frac{2}{9}\\
        \frac{1}{9} & -\frac{1}{3} & \frac{2}{9}\\
        \frac{1}{6} & 0 & -\frac{1}{6}
      \end{pmatrix}.
      $$
      Check that the rows sum to zero as they should.
      \item Let the stationary distribution be given by $\pi = (\pi_0, \pi_1, \pi_2)$, then we have
      $$
      (\pi_0, \pi_1, \pi_2)
      \begin{pmatrix}
        -\frac{1}{3} & \frac{1}{9} & \frac{2}{9}\\
        \frac{1}{9} & -\frac{1}{3} & \frac{2}{9}\\
        \frac{1}{6} & 0 & -\frac{1}{6}
      \end{pmatrix}
      = 0.
      $$
      which yields the equations
      \begin{align}
        \tag{9.1}
        -\frac{1}{3}\pi+0 + \frac{1}{9}\pi_1 + \frac{1}{6}\pi_2 &= 0,\\
        \tag{9.2}
        \frac{1}{9}\pi+0 - \frac{1}{3}\pi_1 &= 0,\\
        \tag{9.3}
        \frac{2}{9}\pi_0 + \frac{2}{9}\pi_1 - \frac{1}{6}\pi_2 &= 0.
      \end{align}
      Solving (say) (9.1) and (9.2) yields
      $$
      \pi_1 = \frac{1}{3}\pi_0,\quad\text{and}\quad \pi_2 = \frac{16}{9}\pi_0
      $$
      Now, using the normalization condition $\pi_0 + \pi_1 + \pi_2 = 1$ we finally get
      $$
      (\pi_0, \pi_1, \pi_2) = (\frac{9}{28}, \frac{3}{28}, \frac{16}{28}).
      $$
      The long-run proportion of time Sam is fit to study is
      $$
      \pi_0 = \frac{9}{28}.
      $$
      \item Sam can work in both states 0 and 1 so, assuming he is supposed to work 7 days a week, his long-run weekly earnings are given by
      \begin{align*}
        \pounds 20 \times 7 \times (\pi_0 + \pi_1) 
        &= \pounds 140 \times \frac{12}{28}\\
        &= \pounds 60.
      \end{align*}
    \end{enumerate}
    \item \textbf{Computer queue}\\
    The interarrival times and runtimes are independent and exponentially distributed. It is the lack-of-memory property which guarantees that $X$ has the Markov property.\\
    The state space is $S = \{0, 1, 2,\ldots\}$ and the generator is 
    $$
    \vb{A} =
    \begin{pmatrix}
      -\lambda & \lambda & 0 & 0 & \ldots\\
      \mu & -(\lambda + \mu) & \lambda & 0\\
      0 & \mu & -(\lambda + \mu) & \lambda & \ldots\\
      \vdots & \vdots & \vdots & \vdots & \ddots
    \end{pmatrix}
    $$
    Assuming $\lambda > 0$ and $\mu > 0$, the chain is irreducible so the ergodic theorem ensures that there is either a unique stationary distribution or no stationary distribution. Notice also that, although the state space is infinite, $\sup\{-A_{ii}\} < \infty$. If there is a stationary distribution $\pi = (\pi_0, \pi_1, \pi_2, \pi_3, \ldots)$, it satisfies 
    $$
    \pi\vb{A} = 0
    $$
    which leads to the equations
    \begin{equation}\tag{9.4}
      -\lambda\pi_0 + \mu\pi_1 = 0
    \end{equation}
    and
    \begin{equation}\tag{9.5}
      \lambda\pi_{j - 1} - (\lambda + \mu)\pi_j + \mu\pi_{j + 1} = 0\quad\text{for $j \geq 1$}.
    \end{equation}
    From (9.4) we have $\pi_1 = \pi_0(\lambda/\mu)$; then, by induction, (9.5) gives $\pi_j = \pi_0(\lambda/\mu)^j$. In order for this to be a stationary we must also have $\sum_{j = 0}^\infty\pi_j = 1$. Since,
    \begin{equation}\tag{9.6}
      \sum_{j = 0}^\infty\left(\frac{\lambda}{\mu}\right)^j = \frac{1}{1 - (\lambda/\mu)},\quad \text{if $\lambda < \mu$}\quad \text{[geometric progression]},
    \end{equation}
    we find that there is a stationary distribution given by
    $$
    \pi_j = \left(1 - \frac{\lambda}{\mu}\right)\left(\frac{\lambda}{\mu}\right)^j,\quad\text{if $\lambda < \mu$}.
    $$
    If $\lambda \geq \mu$, then the sum on the left-hand side of (9.6) does not converge and hence there is no stationary distribution. Physically, we see that $\lambda > \mu$ corresponds to the case where jobs arrive (on average) faster than they can be dealt with and hence pile up on the processor.
  \end{enumerate}
\end{document}