\documentclass[11pt,a4paper]{article}
\usepackage[margin=1in]{geometry}
\usepackage{amsfonts,amsmath,amssymb,suetterl}
\usepackage{lmodern}
\usepackage[T1]{fontenc}
\usepackage{fancyhdr}
\usepackage{float}
\usepackage[utf8]{inputenc}
\usepackage{physics}
\usepackage{fontawesome}
\DeclareUnicodeCharacter{2212}{-}
\usepackage{mathrsfs}
\usepackage[nodisplayskipstretch]{setspace}

\setstretch{1.5}
\setlength{\headheight}{15.2pt}

\pagestyle{fancy}
\fancyfoot[C]{\thepage}
\fancyfoot[L]{Course Organizer: R. J. HARRIS}
\fancyfoot[R]{rosemary.harris@qmul.ac.uk}

% \renewcommand{\footrulewidth}{0pt}
\renewcommand{\headrulewidth}{0pt}
\parindent 0ex
\setlength{\parskip}{1em}

\begin{document}
  \textbf{MTH734U/MTHM012} \hfill \textbf{Mathematical Sciences}\\
  \textbf{Semester B 2010 - 2011} \hfill \textbf{QMUL}
  \begin{center}
    \textbf{\huge Week 9}
  \end{center}
  \hrule \vspace{2mm} \hrule
  \section*{Key Objective:}
  \textit{Know the ergodic theorem for continuous-time Markov chains and be able to find the stationary distributions for simple examples.}

  \section*{Background:}
  You should now be comfortable with all the material in [T+KJ Sec. VI.6. The ergodic theorem for continuous-time Markov chains is not really covered in the book so please make sure to be familiar with the formulation presented in the lecture notes and/or check out more advanced references such as Grimmett and Stirzaker. In terms of application, you need to be able to construct the generator for simple scenarios and then find the corresponding stationary distribution — see examples in [T+K] and the problems below.

  \section*{Problems:}
  \begin{enumerate}
    \item \textbf{On-off process again}\\
    Do [T + K] Problem VI.6.2. We did this problem a few weeks ago within the framework of renewal theory. Now treat each component as a Markov chain and compare your results. Which method do you find easier?
    \item \textbf{Student health}\\
    Sam Lacker is an undergraduate student whose health appears to be a Markov process fluctuating between three states: O (fit) I (minor illness which prevents him studying but does not stop him working at his part-time job)
  \end{enumerate}

\end{document}