\documentclass[11pt,a4paper]{article}
\usepackage[margin=1in]{geometry}
\usepackage{amsfonts,amsmath,amssymb,suetterl}
\usepackage{lmodern}
\usepackage[T1]{fontenc}
\usepackage{fancyhdr}
\usepackage{float}
\usepackage[utf8]{inputenc}

\usepackage{fontawesome}
\DeclareUnicodeCharacter{2212}{-}
\usepackage{mathrsfs}

\usepackage[nodisplayskipstretch]{setspace}

\setstretch{1.5}
\setlength{\headheight}{15.2pt}

\pagestyle{fancy}
\fancyfoot[C]{\thepage}
\fancyfoot[L]{Course Organizer: R. J. HARRIS}
\fancyfoot[R]{rosemary.harris@qmul.ac.uk}

% \renewcommand{\footrulewidth}{0pt}
\renewcommand{\headrulewidth}{0pt}
\parindent 0ex
\setlength{\parskip}{1em}

\begin{document}
  \textbf{MTH734U/MTHM012} \hfill \textbf{Mathematical Sciences}\\
  \textbf{Semester B 2010 - 2011} \hfill \textbf{QMUL}
  \begin{center}
    \textbf{\huge Week 3}
  \end{center}
  \hrule \vspace{2mm} \hrule
  \section*{Key Objective:}
  \textit{Know and apply key results about the asymptotic behaviour of renewal processes including: the elementary renewal theorem and its derivation from the Blackwell renewal theorem; limiting distributions of number of renewal events, age, and ea;cess life.}
  \section*{Background:}
  Please go through the lecture notes together with [T+K] Sec. VII.4. Note that the latter does not cover transient renewal processes which will not form a significant part of the final exam. [However the method used in Problem 3 below is important in other contexts too so please do try it.] Also note that the paragraphs seem to be wrongly ordered on pages 444—445.\par
  Some of the questions this time are "Problems" from the book. As usual the shorter "Exercises" in the book are also worth looking at please ask if you have any problem getting the stated answers.
  \newpage
  \section*{Problems:}
  \begin{enumerate}
    \item \textbf{On-off process}
    \begin{enumerate}
      \item Do [T+K] Problem 4.2,
      \item What is the asymptotic variance in the number of failures up to time $t$?
    \end{enumerate}
    \item \textbf{Light bulbs again}
    \begin{enumerate}
      \item Let the lifetime of a light bulb be distributed with continuous density $f(x)$ and distribution function $F(x)$ (for $x > 0$). The \textit{hazard rate} (or \textit{failure rate}) h(x) is defined so that the conditional probability that a bulb which has survived to time $x$ will fail during the interval $x, x + \Delta x$ is given by $h(x)\Delta x + o(\Delta x)$. Show that
      $$
      h(x) = \frac{f(x)}{1 - F(x)}\quad \text{for $x > 0$}.
      $$
      \item Now do [T+K] Problem 4.3
    \end{enumerate}
    \item \textbf{Transient renewal process}
    \begin{enumerate}
      \item Consider a transient renewal process with a finite total number of renewals $N$. Let $L$ be the time of the last renewal. If $X_1 = \infty$, then $L = 0$; if $X_1 < \infty$, then $L$ is the sum of $X_l$ and the lifetime of the renewal process whose time origin is at $X_l$ and whose successive intervals are $X_2, X_3, \ldots$. Use this approach to argue that
      $$
      E[L] = E[X_11\{X_1 < \infty\}] + E[L]E[1\{X_1 < \infty\}]
      $$
      where 1 is the indicator function. Hence show that
      $$
      E[L] = \frac{1}{1 - F(\infty)}\int_0^\infty[F(\infty) - F(t)]dt.
      $$
      \item Let $W_0, W_1, W_2,\ldots$ be the successive instants at which vehicles pass a fixed point on the road and the "headways" $(X_l, X_2, X_3, \ldots)$ between cars be independent random variables from a distribution with exponential density $f(t) = \alpha e^{-\alpha t}$, $t \geq 0$. If I arrive at time $t = W_0 = 0$ and I need a time gap of length $\tau$ to get across the road, how long must I wait on average?\par 
      \textbf{(Hints available in tutorial on 2nd February, solutions in tutorial on 9th February)}
    \end{enumerate}
  \end{enumerate}
  %
  \newpage
  \textbf{MTH734U/MTHM012} \hfill \textbf{Mathematical Sciences}\\
  \textbf{Semester B 2010 - 2011} \hfill \textbf{QMUL}
  \begin{center}
    \textbf{\huge Solutions for Week 3}
  \end{center}
  \hrule \vspace{2mm} \hrule
  \section*{General comments:}
  For many problems like this the main difficulty is identifying the renewal process; after that it's just the application of standard formulae. You should know all the asymptotic results given in VI1.4 of [T+K] and/or the lecture. Proofs are only needed where we explicitly covered them.
  \section*{Solutions to Problems:}
  \begin{enumerate}
    \item \textbf{On-off process}
    \begin{enumerate}
      \item Here we're interested in the number of failures up to time $t$ so we identify the renewal events as the ends of the repair periods (when the machine returns to service). In other words if $N(t)$ is the number of (complete) repair periods up to time $t$ then $\{N(t), t \geq 0\}$ is a renewal process. Between consecutive renewal events we have an operating period (exponentially distributed with parameter $\alpha$) and a repair period (exponentially distributed with parameter $\beta$). Since the operating and repair times are statistically independent, the interoccurence distribution could be straightforwardly calculated as the convolution of their two distributions. However, actually all we need is the mean and variance which are simply given by the sums
      \begin{align*}
        \mu &= \frac{1}{\alpha} + \frac{1}{\beta}\\
        \sigma^2 &= \frac{1}{\alpha^2} + \frac{1}{\beta^2}
      \end{align*}
      For $\gg$ the expectation $E[N(t)]$ has the approximate form
      \begin{align*}
        M(t)
        &\approx \frac{t}{\mu} + \frac{\sigma^2 - \mu^2}{2\mu^2}\\
        &= \frac{t}{\frac{1}{\alpha} + \frac{1}{\beta}} + \frac{\left(\frac{1}{\alpha^2} + \frac{1}{\beta^2}\right) - \left(\frac{1}{\alpha} + \frac{1}{\beta}\right)^2}{2\left(\frac{1}{\alpha} + \frac{1}{\beta}\right)^2}\\
        &= \frac{\alpha\beta}{\alpha + \beta}t + \frac{\frac{2}{\alpha\beta}}{2\left(\frac{\alpha + \beta}{\alpha\beta}\right)^2}\\
        &= \frac{\alpha \beta}{\alpha + \beta}t - \frac{\alpha\beta}{(\alpha + \beta)^2}.
      \end{align*}
      The astute reader may have two questions:
      \begin{itemize}
        \item \textbf{Question:} Does it matter that we don't know how long the machine has already been operating at time $t = 0$.\\
        \textbf{Answer:} No, because for an exponential interoccurence distribution, the distribution excess life is the same as the distribution of all other lives (memoryless property).
        \item \textbf{Question:} If we really want to count failures then shouldn't we identify the renewal events with the beginnings of the failure periods?\\
        \textbf{Answer:} Perhaps, but then the distribution of $X_1$ would be different to the distribution of all the other $X_k$'s. This is called a "delayed renewal process" and is covered in Week 4. Would you expect this different initial condition to make much difference in the long-time limit?
      \end{itemize}
      \item Another application of a standard formula:
      \begin{align*}
        \lim_{t \to \infty}\frac{\text{Var}[N(t)]}{t}
        &= \frac{\sigma^2}{\mu^3}\\
        &= \frac{\left(\frac{1}{\alpha^2} + \frac{1}{\beta^2}\right)}{\left(\frac{1}{\alpha} + \frac{1}{\beta}\right)^3}\\
        &= \frac{\alpha\beta(\alpha^2 + \beta^2)}{(\alpha + \beta)^3}
      \end{align*}
    \end{enumerate}
    \item \textbf{Light bulbs again}
    \begin{enumerate}
      \item The lifetime of the the bulb is a non-negative continuous random variable $X$. We're interested in the conditional probability that the bulb fails in the interval $(x, x + \Delta x]$ (where $x > 0$) given that it has survived to time $x$. This can be calculated as follows.
      \begin{align*}
        \text{Pr}\{x < X \leq x + \Delta x | x<X\}
        &= \frac{\text{Pr}\{x < X \leq x + \Delta \}}{\text{Pr}\{x < X\}}\\
        &= \frac{f(x)\Delta x}{1 - \text{Pr}\{X \leq x\}} + o(\Delta x)\\
        &= \frac{f(x)\Delta x}{1 - F(x)} + o(\Delta x).
      \end{align*}
      So
      \begin{equation}\tag{3.1}
        h(x) = \frac{(x)}{1 - F(x)}\quad \text{for $x > 0$}.
      \end{equation}
      \item Here we're told that $h(x) = \theta x$ which (assuming $\theta > 0$) implies that the lightbulb gets more likely to fail as time increases (i.e., it wears out). We now need to use (3.1) to obtain the corresponding distribution. This involves solving a first order ODE for $F(x)$ or [equivalently for $f(x) = F^\prime(x)$]. For general $h(x)$ we have
      \begin{align*}
        h(x) &= \frac{F^\prime(x)}{1 - F(x)}\\
        h(x) &= -\frac{d}{dx}\ln[1 - F(x)]
      \end{align*}
      and integrating both sides (with the boundary condition $F(0) = 0$ yields
      $$
      e^{-\int_0^xh(t)dt} = 1 - F(x)
      $$
      i.e,
      $$
      F(x) = 1- e^{-\int_0^xh(x)}dt,\quad f(x) = h(x)e^{-\int_0^xh(t)dt}.
      $$
      For $h(x) = \theta x$ this gives
      $$
      F(x) = 1 - e^{-\theta x^2/2},\quad f(x) = \theta x e^{-\theta x^2/2}
      $$
      defined, of course, for $x > 0$. Standard Gaussian integrations show that this distribution has mean
      $$
      \mu = \sqrt{\frac{\pi}{2\theta}},
      $$
      and variance
      $$
      \sigma^2 = \frac{4 - \pi}{2\theta}.
      $$
      Now we can just use the expression for the asymptotic distribution of age
      \begin{equation}\tag{3.2}
        \lim_{t \to\infty}\text{Pr}\{\delta_t \geq y\} = 1 - \tilde{H}(y)
      \end{equation}
      where
      \begin{equation}\tag{3.3}
        \tilde{H}(y) = \frac{1}{\mu}\int_0^y[1 - F(z)]dz
      \end{equation}
      is the asymptotic distribution of excess life with corresponding density
      $$
      \tilde{h}(y) = \frac{1}{\mu}[1 - F(y)].
      $$
      (I've used the tildes here to distinguish from the $h$ used above for hazard rate.) Note: You should know (without proof) the asymptotic distribution of excess life (3.3) and how to get from there to the limiting distribution of age (3.2)- see pages 443-445 of [T+K]. Equation (3.2) tells us that in the asymptotic limit the excess life and the age have the same distribution which is an intuitively plausible result (why?). From here we see that, for t > 0, the mean age of the light bulb in service is approximately given by
      \begin{align*}
        E[\delta_t]
        &\approx \int_0^\infty y\tilde{h}(y)dy\\
        &= \frac{1}{\mu}\int_0^\infty y[1 - F(y)]dy\\
        &= \frac{1}{\mu}\int_0^\infty y\left\{\int_y^\infty f(t)dt\right\}dy\\
        &= \frac{1}{\mu}\int_0^\infty f(t)\left\{\int_0^t ydy\right\}dt\quad\text{[changing order of integration]}\\
        &= \frac{1}{2\mu}\int_0^\infty t^2f(t)dt\\
        &= \frac{\sigma^2 + \mu^2}{2\mu}\\
        &= \sqrt{\frac{2}{\pi\theta}}.
      \end{align*}
    \end{enumerate}
    \item \textbf{Transient renewal process}
    \begin{enumerate}
      \item We condition, as suggested, on $X_1$. If $X_1 = \infty$, then $L = 0$; if $X_l \infty$, then $L$ is the sum Of $X_1$ and the "lifetime" (meaning, in this context, the time until the last renewal) of the renewal process whose time origin is at $X_l$ and whose successive intervals are $X_2,\, x_3$. But since the $X_k$'s are i.i.d., this second renewal process has the same expected lifetime as the first one. Hence
      \begin{equation}\tag{3.4}
        \begin{aligned}[b]
          E[L]
          &= E[X_11\{X_1 < \infty\}] + E[L]E[1\{X_1 < \infty\}]\\
          &= \int_0^\infty xdF(x) + E[L]F(\infty)\\
          &= \int_0^\infty \left\{\int_0^x dt\right\}dF(x) + E[L]F(\infty)\\
          &= \int_0^\infty\left\{\int_t^\infty dF(x)\right\}dt + E[L]F(\infty)\quad \text{[changing order of integration]}\\
          &= \int_0^\infty [F(\infty) - F(t)]dt + E[L]F(\infty).
        \end{aligned}
      \end{equation}
      If you're worried by the limits when you change the order of integration, then you might find it easier to use indicator variables throughout -cf. the discussion on page 45 of [T+K] for the $F(\infty) = 1$ case. Finally, simple rearrangement of (3.4) gives the required expression:
      \begin{equation}\tag{3.6}
        E[L] = \frac{1}{1 - F(\infty)}\int_0^\infty[F(\infty) - F(t)]dt.
      \end{equation}
      \item Assuming I don't want to get hit by a car, I have to wait until an interoccurence time which is longer than $\tau$. In other words, I will start crossing at $L = VV_n$ if and only if $X_1 \leq \tau,\ldots,X_n\leq \tau$ and $X_{n+l} > \tau$. We see that $L$ is the last renewal in a modified renewal process whose interoccurence times are given by
      $$
      \tilde{X}_n =
      \begin{cases}
        X_n & \text{if $X_n \leq \tau$}\\
        +\infty & \text{otherwise}
      \end{cases}
      $$
      If the distribution of $X_n$ is $F$ then $\tilde{X}_n$ is
      $$
      \tilde{F}(t) =
      \begin{cases}
        F(t) & \text{if $t \leq \tau$}\\
        F(t) & \text{if $t > \tau$}
      \end{cases}
      $$
      and, using (3.6), the expected time to cross the road is
      \begin{align*}
        E[L]
        &= \frac{1}{1 - \tilde{F}(\infty)}\int_0^\infty [\tilde{F}(\infty) - \tilde{F}(t)]dt.\\
        &= \frac{1}{1 - F(\tau)}\int_0^\tau[F(\tau) - F(t)].
      \end{align*}
      In this question, the headways have an exponential density, i.e, $f(t) = \alpha e^{-\alpha t}$ or $F(t) = 1 - e^{-\alpha t}$ which leads straightforwardly to
      $$
      E[L] = \frac{1}{\alpha}(e^{\alpha t} - 1) - \tau.
      $$
      It's worth checking that $E[L] \to 0$ as $\tau \to 0$ and $E[L] \to \infty$ as $\tau \to \infty$.
    \end{enumerate}
  \end{enumerate}
\end{document}