\documentclass[11pt,a4paper]{article}
\usepackage[margin=1in]{geometry}
\usepackage{amsfonts,amsmath,amssymb,suetterl}
\usepackage{lmodern}
\usepackage[T1]{fontenc}
\usepackage{fancyhdr}
\usepackage{float}
\usepackage[utf8]{inputenc}
\usepackage{physics}

\usepackage{fontawesome}
\DeclareUnicodeCharacter{2212}{-}
\usepackage{mathrsfs}

\usepackage[nodisplayskipstretch]{setspace}

\setstretch{1.5}
\setlength{\headheight}{15.2pt}

\pagestyle{fancy}
\fancyfoot[C]{\thepage}
\fancyfoot[L]{Course Organizer: R. J. HARRIS}
\fancyfoot[R]{rosemary.harris@qmul.ac.uk}

% \renewcommand{\footrulewidth}{0pt}
\renewcommand{\headrulewidth}{0pt}
\parindent 0ex
\setlength{\parskip}{1em}

\begin{document}
  \textbf{MTH734U/MTHM012} \hfill \textbf{Mathematical Sciences}\\
  \textbf{Semester B 2010 - 2011} \hfill \textbf{QMUL}
  \begin{center}
    \textbf{\huge Week 4}
  \end{center}
  \hrule \vspace{2mm} \hrule
  \section*{Key Objective:}
  \textit{Understand what is meant by a delayed renewal process and a stationary renewal process. Analyse scenarios involving these and other simple generalizations.}
  \section*{Background:}
  Please read Sec. VII.5 of [T + K]. There are lots of examples there to give you a flavour of the types of situations where renewal processes arise (and hence of the types of questions you might be asked). The key to each problem is to identify what the renewal cycle is.
  \section*{Problems:}
  \begin{enumerate}
    \item \textbf{Poisson process again:}
    \begin{enumerate}
      \item For a pure renewal process $\{N(t,\, t\geq 0)\}$ with exponential interocccurence distribution,
      i.e.,
      \begin{equation}\tag{3.1}
        F(t) = 1 - e^{\lambda t}\quad\text{for $t \geq 0$},
      \end{equation} 
      find the asymptotic distribution of excess life.
      \item Now consider a delayed renewal process with $F$ of (3.1) the distribution of $X_2, X_3,\allowbreak X_4,\ldots$ and a different exponential distribution for $X_1$.
      $$
      G(t) = 1 - e^{-\omega t}\quad \text{for $t \geq 0$}.
      $$
      Calculate $M_D(t) = E[N(t)]$ and discuss its behaviour as $t \to \infty$.
    \end{enumerate}
    \item \textbf{A stationary renewal process}\\
    Consider another delayed renewal process where $F$ (the distribution of $X_2, X_3, X_4,\ldots$) has a gamma distribution
    $$
    F(t) = 1 - e^{-t} - te^{-t},\quad f(t) = te^{-t}\quad \text{for $t \geq 0$}.
    $$
    \begin{enumerate}
      \item What must be the distribution of $X_l$ if this delayed renewal process is to be stationary.
      \item Check that the renewal function is linear in this case.
    \end{enumerate}
    \item \textbf{The lazy professor}\\
    Do [T + k] problems 5.4
    \item \textbf{Aeroplane maintenance}\\
    A vital component of an aeroplane is replaced at a cost $b$ whenever it reaches a certain age $A$. If it fails earlier the cost of replacement is $a$. Given that the lifetime distribution of components of this type is triangular with probability density $f(x) = 2x$ on $(0, 1)$, determine an expression for the long-run replacement cost per unit time.
    \vspace{1cm}
    \begin{center}
      \textbf{(Hints available in tutorial on 8th February, solutions in tutorial on 15th February)}
    \end{center}
  \end{enumerate}
  %
  \newpage
  \textbf{MTH734U/MTHM012} \hfill \textbf{Mathematical Sciences}\\
  \textbf{Semester B 2010 - 2011} \hfill \textbf{QMUL}
  \begin{center}
    \textbf{\huge Solutions for Week 4}
  \end{center}
  \hrule \vspace{2mm} \hrule
  \section*{General comments:}
  It seems that the main problems here are not concepts of renewal theory but technicalities such as integration! Try to develop strategies to check your answers during/after long calculations and thus eliminate silly mistakes.
  \begin{enumerate}
    \item \textbf{Poisson process again}
    \begin{enumerate}
      \item First note that the mean of the exponential interoccurence distribution is
      \begin{align*}
        \mu
        &= \int_0^\infty \{1 - F(t)\}dt\\
        &= \int_0^t e^{-\lambda t}dt\\
        &= \frac{1}{\lambda}.
      \end{align*}
      The asymptotic distribution of excess life (i.e, $\lim_{t \to \infty}\text{Pr}\{\gamma_t \leq x\}$) is given by
      \begin{align*}
        H(x)
        &= \mu^{-1}\int_0^x\{1 - F(t)\}dt\\
        &= \lambda\int_0^xe^{-\lambda t}dt\\
        &= \left[-e^{-\lambda t}\right]_{t = 0}^{t = x}\\
        &= 1 - e^{-\lambda x}.
      \end{align*}
      The fact that the asymptotic distribution of excess life is identical to the interoccurence distribution is, of course, a particular property of the Poisson process.
      \item You can do this directly with Laplace transforms or simply start from the formula derived in lectures
      \begin{equation}\tag{4.1}
        M_D(t) = G*(M(t) + 1)
      \end{equation}
      where $M(t)$ is the renewal function for the corresponding pure process and $*$ denotes the convolution operation. For a pure Poisson process we have
      $$
      M(t) = \lambda t
      $$
      and so, using (4.1) yields
      \begin{align*}
        M_D(t)
        &= \int_0^\infty[M(t - x) + 1]g(x)dx\\
        &= \int_0^t[\lambda(t - x) + 1]\omega e^{-\omega x}dx\\
        &= \int_0^t\lambda (t - x)\omega e^{-\omega x}dx + \int_0^t\omega e^{-\omega x}dx\\
        &= \lambda t + \frac{\lambda}{\omega}e^{-\omega t} - \frac{\lambda}{\omega} + 1 - e^{-\omega t}.
      \end{align*}
      You should check that, for $\omega = \lambda$, $M_D(t) = M(t)$. Furthermore, it's straightforward to show the asymptotic limit
      $$
      \lim_{t\to\infty}\frac{M(t)}{t} = \lambda.
      $$
      In other words the limiting behaviour is not affected by the distribution $G$.
    \end{enumerate}
    \item \textbf{A stationary renewal process}
    \begin{enumerate}
      \item For this to be a stationary renewal process $X_l$ must have the distribution of the excess life, i.e.,
      \begin{align*}
        G(x)
        &= \mu^{-1}\int_0^x\{1 - F(t)\}dt\\
        &= \frac{\int_0^x e^{-t} + te^{-t}dt}{\int_0^\infty e^{-t} + te^{-t}dt}\\
        &= 1 - e^{-x} - \frac{x}{2}e^{-x}\quad\text{[after integration by parts]}
      \end{align*}
      with corresponding density
      $$
      g(x) = \frac{1}{2}e^{-x} + \frac{x}{2}e^{-x}.
      $$
      \item Again we can use
      $$
      M_D(t) = G * (M(t) + 1)
      $$
      where the renewal function for a pure renewal process with gamma density of interoccurence times is (see Week 2 Handout)
      $$
      M(t) = \frac{t}{2} - \frac{1}{4}(1 - e^{2t}).
      $$
      So we have
      \begin{align*}
        M_D(t)
        &= \int_0^t[M(t - x) + 1]g(x)dx\\
        &= \int_0^t\left[\frac{(t - x)}{2} + \frac{3}{4} + \frac{1}{4}e^{-2(t - x)}\right]\left(\frac{1}{2}e^{-x} + \frac{x}{2}e^{-x}\right)dx
      \end{align*}
      Multiplying out the brackets and doing some rather tedious (!) integration by parts finally yields
      $$
      M_D(t) = \frac{t}{2},
      $$
      as required.
    \end{enumerate}
    \item \textbf{The lazy professor}\\
    Let us define the renewal events as the times when the professor replaces both lightbulbs. Sucessive renewal intervals $X_1, X_2, X_3, \ldots$ each contain a portion $(Y_l, Y_2, Y_3, \ldots)$ when the office is fully lit and a corresponding portion when the office is only half lit $(X_l - Y_l, X_2 - Y_2, X_3 - Y_3,\ldots)$. Letting $p(t)$ be the probability that the office is fully lit, renewal theory gives
    \begin{align*}
      \text{Long-run fraction of time office is half-lit}
      &= 1 - \lim_{t\to\infty}p(t)\\
      &= 1 - \frac{E[Y_1]}{E[X_1]}.
    \end{align*}
    Now if the lifetimes of two bulbs are independent random variables, $Z_A$ and $Z_B$ then $E[Y_1]$ is the mean time for the first bulb to fail, i.e., $E[\text{max}\{Z_A, Z_B\}]$ and $E[X_1]$ is the mean time for the second bulb to fail, i.e., $E[\text{max}\{Z_A, Z_B\}]$. There are a variety of methods to calculate these expectations as illustrated below...
    \begin{enumerate}
      \item  Here the lifetimes of the bulbs are exponentially distributed, i.e., $Z_A$ and $Z_B$ have probability density function $f_Z(z) = \alpha e^{-\alpha x}$ for $z \geq 0$. To calculate $E[Z_1]$ we can use any of the
      following methods:
      \begin{enumerate}
        \item First, using brute force and conditional expectations, we write
        $$
        E[Y_1] = E[Z_A|Z_A < Z_B]\text{Pr}\{Z_A < Z_B\} + E[Z_B|Z_A > Z_B]\text{Pr}\{Z_A > Z_b\}
        $$
        which can be expressed using indicator variables as
        \begin{align*}
          E[Y_1]
          &= \int_0^\infty\left\{\int_0^\infty[z_A\vb{1}\{z_A < z_B\} + z_B\vb{1}\{z_A > z_B\}]f_Z(z_A)f_Z(z_B)dz_B\right\}dz_A\\
          &= \int_0^\infty z_A\alpha e^{-\alpha z_A}\left\{\int_{z_A}^\infty \alpha e^{-\alpha z_B}dz_B\right\}dz_A\\
          &\quad + \int_0^\infty \alpha e^{-\alpha z_A}\left\{\int_0^{z_A} z_B\alpha e^{-\alpha z_B}dz_B\right\}dz_A\\
          &= \frac{\alpha}{(2\alpha)^2} + \frac{\alpha}{(2\alpha)^2}\quad \text{[straightforward but tedious integration]}\\
          &= \frac{1}{2\alpha}.
        \end{align*}
        \item An equivalent, but slightly more elegant route is to recall the general result that the joint probability density function of the maximum $X_l$ and minimum $Y_l$ of two independent random variables each with probability density function $f_z$ is given by $2f_Z(x)f_Z(y),\, 0 < y < x < \infty$ which is almost obvious if you think about it long enough! Using this, we have
        \begin{align*}
          E[Y_1]
          &= 2\int_0^\infty f_Z(x)\left\{\int_0^x yf_Z(y)dy\right\}dx\\
          &= 2\int_0^\infty \alpha e^{-\alpha x}\left\{\int_0^x y\alpha e^{-\alpha y}dy\right\}dx\\
          &= 2\frac{\alpha}{(2\alpha)^2}\\
          &= \frac{1}{2\alpha}.
        \end{align*}
        \item Alternatively, one can easily show that:
        \begin{align*}
          \text{Pr}\{Y > y\}
          &= \text{Pr}\{min\{Z_A, Z_B\} > y\}\\
          &= \text{Pr}\{Z_A > y,\, Z_B > y\}\\
          &= \text{Pr}\{Z_A > y\}\times \text{Pr}\{Z_B > y\}\quad\text{independence}\\
          &= e^{-\alpha y}e^{-\alpha y}\\
          &= e^{-2\alpha y}
        \end{align*}
        Hence we see that $Y_l$ is exponentially distributed with parameter 2a and the result for its mean therefore follows.
      \end{enumerate}
      One can do analogous calculations to find $E[X_1]$ but in fact, in this case, it is easier simply to observe that after the first bulb has failed, the second bulb will have an average excess lifetime of $1/\alpha$ (memoryless property of exponential distribution). We therefore see that the average interval between ladder uses is
      \begin{align*}
        E[X_1]
        &= \frac{1}{2\alpha} + \frac{1}{\alpha}\\
        &= \frac{3}{2\alpha}
      \end{align*}
      and conclude that
      \begin{align*}
        \text{Long-run fraction of time office is half lit} 
        &= 1 - \frac{E[Y_1]}{E[X_1]}\\
        &= \frac{2}{3}.
      \end{align*}
      \item Now we need to work out $E[Y_1]$ and $E[X_1]$ when the bulb lifetimes have a uniform $(0, 1)$ distribution. Using method 2 above, we have that $f_{X_1, Y_1}(x, y) = 2$, $0 < y < x < 1$ and can carry out the relevant double integrations to calculate $E[X_1]$ and $E[Y_1]$. Alternatively, we can avoid worrying about the limits on double integrals by following the analogue of method 3:
      \begin{align*}
        E[Y_1]
        &= \int_0^1 \text{Pr}\{Y_1 > y\}dy\quad \text{[tail probabilities]}\\
        &= \int_0^1\text{Pr}\{Z_A > y\}\text{Pr}\{Z_B > y\}dy\\
        &= \int_0^1(1 - y)^2dy\\
        &= \left[-\frac{1}{3}(1 - y)^3\right]_{y = 0}^{y = 1}\\
        &= \frac{1}{3}.
      \end{align*}
      Similarly,
      \begin{align*}
        E[X_1]
        &= \int_0^1 \text{Pr}\{X_1 > x\}dx\\
        &= \int_0^1(1 - \text{Pr}\{X_1 < x\})dx\\
        &= \int_0^1(1 - \text{Pr}\{Z_A < x\}\text{Pr}\{X_B < x\})dx\\
        &= \int_0^1(1 - x^2)dx\\
        &= \left[x - \frac{1}{3}x^3\right]_{x = 0}^{x = 1}\\
        &= \frac{2}{3}.
      \end{align*}
      Hence we can conclude
      \begin{align*}
        \text{Long-run fraction of time office is half lit}
        &= 1 - \frac{E[Y]}{E[X]}\\
        &= \frac{1}{2}.
      \end{align*}
    \end{enumerate}
    \item \text{Aeroplane maintenance}\\
    Let $X_l,\, X_2,\, \ldots$ be the runtimes (i.e., the durations in service) of the component and its replacements. The $X_k$ are assumed independent with common distribution function
    $$
    \tilde{F} =
    \begin{cases}
      F(x) & \text{if $x < A$}\\
      1 & \text{if $x \geq A$}
    \end{cases}
    $$
    where $F$ is the distribution function of the usual lifetime (i.e., without replacement). Hence, using our favourite tail-probabilities trick,
    $$
    E[X_1] = \int_0^\infty[1 - \tilde{F}(x)]dx = \int_0^A[1 - F(x)]dx
    $$
    Now let $Y_l,\, Y_2,\, \ldots$ be the cost incurred by the $k$th replacement. Then
    $$
    Y_k =
    \begin{cases}
      a & \text{with probability $F(A)$}\\
      b & \text{with probability $1 - F(A)$}
    \end{cases}
    $$
    and
    $$
    E[Y_1] = aF(A) + b[1 - F(A)].
    $$
    So using the so-called "renewal-reward theorem" we obtain the general result:
    $$
    \text{Long-run replacement cos} = \frac{E[Y_1]}{E[X_1]} = \frac{aF(A) + b[1 - F(A)]}{\int_0^\infty[1 - F(x)]dx}.
    $$
    In this particular case
    $$
    F(x) = \int_0^x f(u)du\int_0^x 2udu = x^2\quad \text{for $0 < x < 1$}.
    $$
    Hence, assuming $A \leq 1$, we have
    \begin{align*}
      \text{Long-run replacement cost}
      &= \frac{aA^2 + b[1 - A^2]}{\int_0^A[1 - x^2]dx}\\
      &= \frac{aA^2 + b[1 - A^2]}{A - A^3/3}\\
      &= \frac{b + (a - b)A^2}{A(1 - A^2/3)}.
    \end{align*}
    Note that, in the trivial case where $A > 1$, we have
    \begin{align*}
      \text{Long-run replacement cost}
      &= \frac{a}{\int_0^1[1 - x^2]dx}\\
      &= \frac{a}{2/3}\\
      &= \frac{3a}{2}.
    \end{align*}
  \end{enumerate}
\end{document}