\documentclass[11pt,a4paper]{article}
\usepackage[margin=1in]{geometry}
\usepackage{amsfonts,amsmath,amssymb,suetterl}
\usepackage{lmodern}
\usepackage[T1]{fontenc}
\usepackage{fancyhdr}
\usepackage{float}
\usepackage[utf8]{inputenc}
\usepackage{physics}
\usepackage{fontawesome}
\DeclareUnicodeCharacter{2212}{-}
\usepackage{mathrsfs}
\usepackage[nodisplayskipstretch]{setspace}

\setstretch{1.5}
\setlength{\headheight}{15.2pt}

\pagestyle{fancy}
\fancyfoot[C]{\thepage}
\fancyfoot[L]{Course Organizer: R. J. HARRIS}
\fancyfoot[R]{rosemary.harris@qmul.ac.uk}

% \renewcommand{\footrulewidth}{0pt}
\renewcommand{\headrulewidth}{0pt}
\parindent 0ex
\setlength{\parskip}{1em}
\allowdisplaybreaks

\begin{document}
  \textbf{MTH734U/MTHM012} \hfill \textbf{Mathematical Sciences}\\
  \textbf{Semester B 2010 - 2011} \hfill \textbf{QMUL}
  \begin{center}
    \textbf{\huge Week 11}
  \end{center}
  \hrule \vspace{2mm} \hrule

  \section*{Key Objective:}
  \textit{Understand the reflection principle and be able to use it to calculate the time to first reach a level and the zeros of Brownian motion}.

  \section*{Background:}
  Please read [T+K] Secs. VIII 2.1—3.2 and make sure you understand the reflection principle and how to use it.
  \newpage

  \section*{Problems:}
  \begin{enumerate}
    \item \textbf{Trigonometric identities}\\
    Complete the proof Of the result for $\theta(t, t + s)$ from the lecture by showing that
    $$
    \arctan\sqrt{\frac{s}{t}} = \arccos\sqrt{\frac{t}{s + t}}.
    $$
    \item \textbf{Zeros (or lack of them)}\\
    Do [T + K] Problems V111.2.1 and V111.2.2.
    \item \textbf{Probability distributions}\\
    Do [T+K] Problem V111.2.3
    \item \textbf{More reflection principle}\\
    Do [T+K] Problem VIII.2.4 — part of this also featured on the 2005 exam paper.\par 
    [Note that the question in the book contains an obvious typo (the "m" at the very bottom of page 497 should be a "z"); don't forget to answer the part overleaf as well...]
  \end{enumerate}
  \textbf{(Hints available in tutorial on 30th March, solutions posted online about 6th April)}
  \newpage

  \textbf{MTH734U/MTHM012} \hfill \textbf{Mathematical Sciences}\\
  \textbf{Semester B 2010 - 2011} \hfill \textbf{QMUL}
  \begin{center}
    \textbf{\huge Solutions for Week 11}
  \end{center}
  \hrule \vspace{2mm} \hrule

  \section*{General comments:}

\end{document}