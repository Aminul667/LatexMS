\documentclass[11pt,a4paper]{article}
\usepackage[margin=1in]{geometry}
\usepackage{amsfonts,amsmath,amssymb,suetterl}
\usepackage{lmodern}
\usepackage[T1]{fontenc}
\usepackage{fancyhdr}
\usepackage{float}
\usepackage[utf8]{inputenc}
\usepackage{physics}
\usepackage{fontawesome}
\DeclareUnicodeCharacter{2212}{-}
\usepackage{mathrsfs}
\usepackage[nodisplayskipstretch]{setspace}

\setstretch{1.5}
\setlength{\headheight}{15.2pt}

\pagestyle{fancy}
\fancyfoot[C]{\thepage}
\fancyfoot[L]{Course Organizer: R. J. HARRIS}
\fancyfoot[R]{rosemary.harris@qmul.ac.uk}

% \renewcommand{\footrulewidth}{0pt}
\renewcommand{\headrulewidth}{0pt}
\parindent 0ex
\setlength{\parskip}{1em}

\begin{document}
  \textbf{MTH734U/MTHM012} \hfill \textbf{Mathematical Sciences}\\
  \textbf{Semester B 2010 - 2011} \hfill \textbf{QMUL}
  \begin{center}
    \textbf{\huge Week 5}
  \end{center}
  \hrule \vspace{2mm} \hrule
  \section*{Key Objective:}
  \textit{In the case of a discrete renewal process be able to use the renewal argument to derive the equation satisfied by the renewal function. Understand the form and solution of the general renewal equation.}
  
  \section*{Background:}
  Please read Sec. VII.6 - VII.6.1 of [T+K] - bonus points for spotting the typos. (Ignore the population growth stuff in VII.6.2.) Discrete renewal processes are simply a special case of the general model discussed in previous weeks. However, because they are somewhat simpler, we delve a bit further into the formalism of renewal theory. As an example application, it's also worth (re)reading the block replacement example back in V11.2.2.\par
  This week marks the end of Section A so please use any spare time to consolidate all the material on renewal phenomena. I've included two relatively straightforward exam questions overleaf; some more interesting questions (for the case of continuous lifetimes) with slightly different notation can be found on the papers from 2009 and 2010.

  \section*{Problems:}
  \begin{enumerate}
    \item \textbf{Geometric distribution of lifetimes}\\
    Do [T + K] Problem 6.1.
    \item \textbf{Throwing a die}\\
    Do [T + K] Problem 6.2.
    \item \textbf{Question 5 from 2003 exam paper}\\
    See overleaf.
    \item \textbf{Question 6 from 2003 exam paper}\\
    See overleaf.\par 
    \textbf{(Hints available in tutorial on 16th February, solutions in tutorial on 23rd February)}
    \item Let $\{N(t),\, t \geq 0\}$ be a continuous time renewal process with interarrival times $X_i$ which are independent and identically distributed. Let $W_n = X_1 + X_2 +\ldots + X_n$ be the waiting time until the occurrence of the nth event. Let $M(t) = \mathbb{E}N(t)$.
    \begin{enumerate}
      \item (\textit{6 marks}) Prove that
      $$
      M(t) = \sum_{n = 1}^\infty F_n(t)
      $$
      where $F_n(t) = \mathbb{P}(W_n \leq t)$.
      \item (\textit{14 marks}) Prove that
      $$
      \mathbb{E}W_{N(t) + 1} = \mathbb{E}[X_1](M(t) + 1).
      $$ 
    \end{enumerate}
    \item Let $\{N(n), n\geq 0\}$ be a discrete time renewal process with interarrival times $X_i$ which are independent and identically distributed with distribution $\mathbb{P}(X_i = k) = p_k$ for k 2 1. and cumulative distribution function $F_x(n) = \mathbb{P}(X_i \leq n)$.
    \begin{enumerate}
      \item \textit{(14 marks)} Prove that the renewal function $M(n) = \mathbb{E}N(n)$ solves the equation
      $$
      M(n) = F_X(n) + \sum_{k = 1}^{n - 1}p_kM(n - k)
      $$
      \item \textit{(6 marks)} Suppose that $p_1 = 0.1,\, p_2 = 0.2,\, p_3 = 0.6$, and$p_4 = 0.1$. Determine $M(n)$ for $n = 1,\, 2,\, 3$. Determine $\lim_{n\to\infty} \frac{M(n)}{n}$.
    \end{enumerate}
  \end{enumerate}
  %
  \newpage
  \textbf{MTH734U/MTHM012} \hfill \textbf{Mathematical Sciences}\\
  \textbf{Semester B 2010 - 2011} \hfill \textbf{QMUL}
  \begin{center}
    \textbf{\huge Solutions for Week 5}
  \end{center}
  \hrule \vspace{2mm} \hrule
  \section*{General comments:}
  The crucial step in discrete renewal theory is to condition on the first life Xl to construct a renewal equation, such as (5.1), (5.3) and (5.9) below. You should make sure you are happy with this type of "renewal argument".

  \section*{Solutions to Problems:}
  \begin{enumerate}
    \item \textbf{Geometric distribution of lifetimes}\\
    In this question, the lifrtimes $X_1, X_2,\ldots$ have the Geometric distribution, i.e.,
    $$
    p_k = \text{Pr}\{X_1 = k\} = \alpha(1 - \alpha)^{k - 1}\quad \text{for $k = 1, 2, \ldots$},
    $$
    where $0 < a < 1$. Notice that $p_0 = 0$ so that all the lifetimes are strictly positive. (This makes things much easier since at most one renewal can occur in any time period.)
    \begin{enumerate}
      \item To calculate $\{u_n\}$ we need to solve the particular renewal equation
      \begin{equation}\tag{5.1}
        u_n = \delta_n + \sum_{k = 0}^np_ku_{n - k}\quad \text{for $n = 0.1.\ldots$}
      \end{equation}
      where
      $$
      \delta_n =
      \begin{cases}
        1 & \text{for $n = 0$},\\
        0 & \text{for n > 0}
      \end{cases}
      $$
      [Note that "determine" in the question implies you can simply quote (5.1) but, of course, you should know how to derive it - see [T+K] page 459.1]\\
      The renewal equation is easy to solve recursively. First note that for $n = 0$ we get $u_0 = 1$. This is true for any probability distribution with $p_0 = 0$; it's simply because we are counting the initial renewal which occurs at $t = 0$. Armed with knowledge of $u_0$, we can calculate
      \begin{align*}
        u_1
        &= p_1u_0\\
        &= \alpha \times 1\\
        &= \alpha.
      \end{align*}
      Similarly,
      \begin{align*}
        u_2
        &= p_1u_1 + p_2u_2\\
        &= \alpha^2 + \alpha(1 - \alpha)\\
        &= \alpha.
      \end{align*}
      Direct calculation also yields $u_3 = \alpha$, suggesting the proposition
      \begin{equation}\tag{5.2}
        u_n = \alpha\quad \text{for $n \geq 1$}.
      \end{equation}
      Indeed this is true as is easily proven by strong induction:
      \begin{itemize}
        \item Base case is $n = 1$ (already checked).
        \item Assume proposition is true for $1 \leq n \leq m$ and consider the $n = m + 1$ case:
        \begin{align*}
          u_{m + 1}
          &= \sum_{k = 0}^{m + 1}p_ku_{m + 1 - k}\\
          &= \sum_{k = 1}^{m + 1}p_ku_{m + 1 - k}\quad \text{[since $p_0 = 0$]}\\
          &= p_{m + 1} + \sum_{k = 1}^mp_k\alpha\quad \text{[using induction hypothesis]}\\
          &= \alpha(1 - \alpha)^m + \sum_{k = 1}^m\alpha^2(1 - \alpha)^{k - 1}\\
          &= \alpha(1 - \alpha)^m + \alpha^2\left[\frac{1 - (1 - \alpha)^m}{1 - (1 - \alpha)}\right]\quad \text{[geometric series]}\\
          &= \alpha(1 - \alpha)^m + \alpha[1 - (1 - \alpha)^m]\\
          &= \alpha.
        \end{align*}
        So, if the proposition is true for $1 \leq n \leq m$, it is also true for $n = m + 1$.
        \item Hence, by induction, the proposition (5.2) is true for all integer $n \geq 1$. 
      \end{itemize}
      The geometric distribution can be thought of as the discrete analogue of the exponential distribution. The probability for renewal in a given time period is constant and $M(n) = \alpha n$ an for all integer $n$.
      \item Let $f_n(m) = \text{Pr}\{\gamma_n = m\}$. As indicated in the question this satisfies the equation
      \begin{equation}\tag{5.3}
        f_n(m) = p_{m + n} + \sum_{k = 0}^{n}f_{n - k}(m)p_k
      \end{equation}
      Equation (5.3) has the form of the general renewal equation with $b_n = p_{m + n}$. Hence by the given Lemma (6.1 in the book) the solution is
      \begin{align*}
        f_n(m)
        &= \sum_{k = 0}^n p_{m + n - k}u_k\\
        &= p_{m + n} + \alpha\sum_{k = 1}^n p_{m + n - k}\quad \text{[using results of part (a)]}\\
        &= p_{m + n} + \alpha\sum_{u = 0}^{n - 1}p_{m + u}\quad \text{[change variables $u = n - k$]}\\
        &= \alpha(1 - \alpha)^{m + n -1} + \alpha\sum_{u = 0}^{n - 1}\alpha(1 - \alpha)^{m + u - 1}\\
        &= \alpha(1 - \alpha)^{m + n -1} + \alpha^2(1 - \alpha)^{m - 1}\sum_{u = 0}^{n - 1}(1 - \alpha)^u\\
        &= \alpha(1 - \alpha)^{m + n - 1} + \alpha(1 - \alpha)^{m - 1}[1 - (1 - \alpha)^n]\\
        &= \alpha(1 - \alpha)^{m - 1}.
      \end{align*}
      Notice that the distribution Of excess life is the same as the interoccurrence distribution ("memoryless" property). In this respect again, the geometric distribution is the discrete analogue of the exponential distribution.
    \end{enumerate}
    \item \textbf{Throwing a die}\\
    This is a seemingly innocuous question that rapidly descends into unpleasant algebra, the pain of which can be justifiably alleviated with Maple or similar. The crucial first step is to find the underlying renewal process we identify the renewal events with tosses of the die and the number on each toss with the associated "lifetime".
    \begin{enumerate}
      \item Within the renewal picture discussed above we're interested in the excess life at $n = 10$, specifically $\text{Pr}\{\gamma_{10} = 3\}$ (i.e., the probability that Marlene stops on a total of 13). just as in the previous question, $f_n(m) = \text{Pr}\{\gamma_n = m\}$, satisfies the renewal equation
      \begin{equation}\tag{5.4}
        f_n(m) = p_{m + n} + \sum_{k = 0}^n f_{n - k}(m)p_k.
      \end{equation}
      where
      $$
      p_n =
      \begin{cases}
        \frac{1}{6} & \text{for $1 \leq n \leq 6$}\\
        0 & \text{otherwise}
      \end{cases}.
      $$
      One could, of course, solve (5.4) by direct recursion to find $f_n(m)$ but it's (slightly) easier to first find the solution un for the particular renewal equation [see (5.1) above]. Setting $p = 1/6$ for convenience one easily finds,
      \begin{align*}
        u_0 &= 1\\
        u_1 &= p\\
        u_2 &= p + p^2
      \end{align*}
      Do not be fooled into thinking that un is a geometric progression. Going one step further, we see
      $$
      u_3 = p + 2p^2 + p^3
      $$
      and the general solution is, in fact,
      $$
      u_n = p(1 + p)^{n - 1}\quad \text{for $1 \leq n \leq 6$},
      $$
      which can be proved by induction or checked via brute force (or Maple!). Continuing the recursion then yields:
      \begin{align*}
        u_7 &= p(1 + p)^6 - p\\
        u_8 &= p(1 + p)^7 - p(1 + 2p)\\
        u_9 &= p(1 + p)^8 - p(1 + p)(1 + 3p)\\
        u_{10} & = p(1 + p)^9 - p(1 + p)^2(1 + 4p)
      \end{align*}
      Can you spot the developing pattern?\\
      Finally, we use the general result (see previous question)
      $$
      f_n(m) = \sum_{k = 0}^np_{m + n - k}u_k
      $$
      to find
      \begin{align*}
        f_{10}(3)
        &= \sum_{k = 0}^{10}p_{13 - k}u_k\\
        &= p(u_7 + u_8 + u _9 + u_{10})\\
        &= p^3(1 + p)(18 + 34p + 35p^2 + 21p^3 + 7p^4 + p^5)\\
        &= \frac{65990113}{362797056}\\
        &\approx 0.1819.
      \end{align*}
      \item This is the easy bit! We have from the discrete renewal theorem (see [T + K], pages 462-463
      \begin{align*}
        \lim_{n\to \infty}\text{Pr}\{\gamma_n = 3\}
        &= \frac{\text{Pr}\{X_1 \geq 3\}}{E[X_1]}\\
        &= \frac{p_3 + p_4 + p_5 + p_6}{p_1 + 2p_2 + 3p_3 + 4p_4 + 5p_5 + 6p_6}\\
        &= \frac{4p}{p(1 + 2 + 3 + 4 + 5 + 6)}\\
        &= \frac{4}{21}\\
        &\approx 0.1905
      \end{align*}
    \end{enumerate}
    \item \textbf{Question 5 from 2003 exam paper}
    \begin{enumerate}
      \item This is standard bookwork (see, e.g., [T + K] 421-422) but make sure you understand all the steps. We start from the "fundamental equivalence"
      \begin{equation}\tag{5.5}
        N(t) \geq k\quad \text{if and only if $W_k \leq t$}
      \end{equation}
      Then (using the notation of the equation)
      \begin{equation}\tag{5.6}
        \begin{aligned}[b]
          M(t)
          &= \mathbb{E}N(t)\\
          &= \sum_{n = 1}^\infty\mathbb{P}(N(t)\geq n)\quad \text{[summing tail probabilities]}\\
          &= \sum_{n = 1}^\infty \mathbb{P}(W_n \leq t)\quad \text{[using 5.5]}\\
          &= \sum_{n = 1}^\infty F_n(t).
        \end{aligned}
      \end{equation}
      \item Again this is bookwork. Here I follow [T+K] (pages 423—424) in doing it in two separate steps but it's also possible to combine the steps. First I introduce indicator random variables (denoted by 1) and evaluate
      \begin{equation}\tag{5.7}
        \begin{aligned}[b]
          \mathbb{E}[X_j\vb{1}\{X_1 + \ldots + X_{j - 1}\leq t\}]
          &= \mathbb{E}[X_j]\mathbb{E}[\vb{1}\{X_1 + \ldots + X_{j - 1} \leq t\}]\quad \text{[independence]}\\
          &= \mathbb{E}[X_j]\mathbb{P}(X_1 + \ldots + X_{j - 1} \leq t)\\
          &= \mathbb{E}[X_1]F_{j - 1}(t).\quad \text{[$X_j$ are indentically distributed]}
        \end{aligned}
      \end{equation}
      Once this is done, the rest is straightforward:
      \begin{align*}
        \mathbb{E}W_{N(t) + 1}
        &= \mathbb{E}[X_1 + \ldots + X_{N(t) + 1}]\\
        &= \mathbb{E}[X_1] + \mathbb{E}\left[\sum_{j = 2}^{N(t) + 1}X_j\right]\\
        &= \mathbb{E}[X_1] + \mathbb{E}\left[\sum_{j = 2}^\infty X_j\vb{1}\{N(t) + 1 \geq j\}\right]\\
        &= \mathbb{E}[X_1] + \sum_{j = 2}^\infty\mathbb{E}[X_j\vb{1}\{X_1 + \ldots + X_{j - 1} \leq t\}]\\
        &= \mathbb{E}[X_1] + \sum_{j = 2}^\infty\mathbb{E}[X_1]F_{j - 1}(t)\quad \text{[using 5.7]}\\
        &= \mathbb{E}[X_1](1 + M(t)).\quad \text{[using 5.6]}
      \end{align*}
      Aside: This result is not obvious since $N(t) + 1$ is not independent of the summands. The trick is to to use the indicator function to get rid of the $N(t) + 1$ in the upper limit of the sum. After that it's safe to exchange the order of the expectation and the sum. This is similar to introducing indicator functions to facilitate exchanging the order of integrals.
    \end{enumerate}
    \item \textbf{Question 6 from 2003 exam paper}
    \begin{enumerate}
      \item This time we have to explicitly derive the renewal equation. We condition on the first renewal event. If this occurs after time n there are obviously no renewals during periods $[1, 2,\ldots n]$. On the other hand if it occurs at time $k \leq n$ then the number of renewals is one plus the average number that occur in the remaining $n - k$ periods. In other words,
      \begin{equation}\tag{5.8}
        \mathbb{E}[N(n) | X_1 = k] =
        \begin{cases}
          0 & \text{if $k > n$},\\
          1 + \mathbb{E}N(n - k) & \text{if $k \leq n$}.
        \end{cases}
      \end{equation}
      Using the law of total probability then yields
      \begin{equation}\tag{5.9}
        \begin{aligned}[b]
          M(n)
          &= \mathbb{E}N(n)\\
          &= \sum_{k = 0}^\infty\text{Pr}\{X_1 = k\}\mathbb{E}[N(n) | X_1 = k]\\
          &= \sum_{k = n + 1}^\infty p_k\times 0 + \sum_{k = 0}^n p_k[1 + M(n - k)]\quad \text{[from (5.8)]}\\
          &= \sum_{k = 1}^n p_k + \sum_{k = 1}^np_kM(n - k)\quad \text{[using that $p_0 = 0$]}\\
          &= F_X(n) + \sum_{k = 1}^{n - 1}p_kM(n - k)\quad \text{[since M(0) = 0]}
        \end{aligned}
      \end{equation}
      as required
      \item Now it's just a case of plugging numbers into (5.9).
      \begin{align*}
        M(1)
        &= F_X(1)\\
        &= p_1\\
        &= 0.1,
      \end{align*}
      \begin{align*}
        M(2)
        &= F_X(2) + p_1M(1)\\
        &= (p_1 + p_2) + p_1M(1)\\
        &= (0.1 + 0.2) + 0.1\times 0.1\\
        &= 0.31
      \end{align*}
      and
      \begin{align*}
        M(3)
        &= F_X(3) + p_1M(2) + p_2M(1)\\
        &= (p_1 + p_2 + p_3) + p_1M(2) + p_2M(1)\\
        &= (0.1 + 0.2 + 0.6) + 0.1\times 0.31 + 0.2\times 0.1\\
        &= 0.951.
      \end{align*}
      Finally, by the discrete renewal theorem,
      \begin{align*}
        \lim_{n\to \infty}\frac{M(n)}{n}
        &= \frac{1}{\mathbb{E}[X_1]}\\
        &= \frac{1}{\sum_{k = 0}^\infty kp_k}\\
        &= \frac{1}{1\times 0.1 + 2\times 0.2 + 3\times 0.6 + 4\times 0.1}\\
        &= \frac{1}{2.7}\\
        &\approx 0.370.
      \end{align*}
    \end{enumerate}
  \end{enumerate}
\end{document}