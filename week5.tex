\documentclass[11pt,a4paper]{article}
\usepackage[margin=1in]{geometry}
\usepackage{amsfonts,amsmath,amssymb,suetterl}
\usepackage{lmodern}
\usepackage[T1]{fontenc}
\usepackage{fancyhdr}
\usepackage{float}
\usepackage[utf8]{inputenc}
\usepackage{physics}
\usepackage{fontawesome}
\DeclareUnicodeCharacter{2212}{-}
\usepackage{mathrsfs}
\usepackage[nodisplayskipstretch]{setspace}

\setstretch{1.5}
\setlength{\headheight}{15.2pt}

\pagestyle{fancy}
\fancyfoot[C]{\thepage}
\fancyfoot[L]{Course Organizer: R. J. HARRIS}
\fancyfoot[R]{rosemary.harris@qmul.ac.uk}

% \renewcommand{\footrulewidth}{0pt}
\renewcommand{\headrulewidth}{0pt}
\parindent 0ex
\setlength{\parskip}{1em}

\begin{document}
  \textbf{MTH734U/MTHM012} \hfill \textbf{Mathematical Sciences}\\
  \textbf{Semester B 2010 - 2011} \hfill \textbf{QMUL}
  \begin{center}
    \textbf{\huge Week 5}
  \end{center}
  \hrule \vspace{2mm} \hrule
  \section*{Key Objective:}
  \textit{In the case of a discrete renewal process be able to use the renewal argument to derive the equation satisfied by the renewal function. Understand the form and solution of the general renewal equation.}
  
  \section*{Background:}
  Please read Sec. VII.6 - VII.6.1 of [T+K] - bonus points for spotting the typos. (Ignore the population growth stuff in VII.6.2.) Discrete renewal processes are simply a special case of the general model discussed in previous weeks. However, because they are somewhat simpler, we delve a bit further into the formalism of renewal theory. As an example application, it's also worth (re)reading the block replacement example back in V11.2.2.\par
  This week marks the end of Section A so please use any spare time to consolidate all the material on renewal phenomena. I've included two relatively straightforward exam questions overleaf; some more interesting questions (for the case of continuous lifetimes) with slightly different notation can be found on the papers from 2009 and 2010.

  \section*{Problems:}
  \begin{enumerate}
    \item \textbf{Geometric distribution of lifetimes}\\
    Do [T + K] Problem 6.1.
    \item \textbf{Throwing a die}\\
    Do [T + K] Problem 6.2.
    \item \textbf{Question 5 from 2003 exam paper}\\
    See overleaf.
    \item \textbf{Question 6 from 2003 exam paper}\\
    See overleaf.\par 
    \textbf{(Hints available in tutorial on 16th February, solutions in tutorial on 23rd February)}
    \item Let $\{N(t),\, t \geq 0\}$ be a continuous time renewal process with interarrival times $X_i$ which are independent and identically distributed. Let $W_n = X_1 + X_2 +\ldots + X_n$ be the waiting time until the occurrence of the nth event. Let $M(t) = \mathbb{E}N(t)$.
    \begin{enumerate}
      \item (\textit{6 marks}) Prove that
      $$
      M(t) = \sum_{n = 1}^\infty F_n(t)
      $$
      where $F_n(t) = \mathbb{P}(W_n \leq t)$.
      \item (\textit{14 marks}) Prove that
      $$
      \mathbb{E}W_{N(t) + 1} = \mathbb{E}[X_1](M(t) + 1).
      $$ 
    \end{enumerate}
    \item Let $\{N(n), n\geq 0\}$ be a discrete time renewal process with interarrival times $X_i$ which are independent and identically distributed with distribution $\mathbb{P}(X_i = k) = p_k$ for k 2 1. and cumulative distribution function $F_x(n) = \mathbb{P}(X_i \leq n)$.
    \begin{enumerate}
      \item \textit{(14 marks)} Prove that the renewal function $M(n) = \mathbb{E}N(n)$ solves the equation
      $$
      M(n) = F_X(n) + \sum_{k = 1}^{n - 1}p_kM(n - k)
      $$
      \item \textit{(6 marks)} Suppose that $p_1 = 0.1,\, p_2 = 0.2,\, p_3 = 0.6$, and$p_4 = 0.1$. Determine $M(n)$ for $n = 1,\, 2,\, 3$. Determine $\lim_{n\to\infty} \frac{M(n)}{n}$.
    \end{enumerate}
  \end{enumerate}
  %
  \newpage
  \section*{General comments:}
  The crucial step in discrete renewal theory is to condition on the first life Xl to construct a renewal equation, such as (5.1), (5.3) and (5.9) below. You should make sure you are happy with this type of "renewal argument".

  \section*{Solutions to Problems:}
  \begin{enumerate}
    \item \textbf{Geometric distribution of lifetimes}\\
    In this question, the lifrtimes $X_1, X_2,\ldots$ have the Geometric distribution, i.e.,
    $$
    p_k = \text{Pr}\{X_1 = k\} = \alpha(1 - \alpha)^{k - 1}\quad \text{for $k = 1, 2, \ldots$},
    $$
    where $0 < a < 1$. Notice that $p_0 = 0$ so that all the lifetimes are strictly positive. (This makes things much easier since at most one renewal can occur in any time period.)
    \begin{enumerate}
      \item To calculate $\{u_n\}$ we need to solve the particular renewal equation
      \begin{equation}\tag{5.1}
        u_n = \delta_n + \sum_{k = 0}^np_ku_{n - k}\quad \text{for $n = 0.1.\ldots$}
      \end{equation}
      where
      $$
      \delta_n =
      \begin{cases}
        1 & \text{for $n = 0$},\\
        0 & \text{for n > 0}
      \end{cases}
      $$
      [Note that "determine" in the question implies you can simply quote (5.1) but, of course, you should know how to derive it - see [T+K] page 459.1]\\
      The renewal equation is easy to solve recursively. First note that for $n = 0$ we get $u_0 = 1$. This is true for any probability distribution with $p_0 = 0$; it's simply because we are counting the initial renewal which occurs at $t = 0$. Armed with knowledge of $u_0$, we can calculate
      \begin{align*}
        u_1
        &= p_1u_0\\
        &= \alpha \times 1\\
        &= \alpha.
      \end{align*}
      Similarly,
      \begin{align*}
        u_2
        &= p_1u_1 + p_2u_2\\
        &= \alpha^2 + \alpha(1 - \alpha)\\
        &= \alpha.
      \end{align*}
      Direct calculation also yields $u_3 = \alpha$, suggesting the proposition
      \begin{equation}\tag{5.2}
        u_n = \alpha\quad \text{for $n \geq 1$}.
      \end{equation}
      Indeed this is true as is easily proven by strong induction:
      \begin{itemize}
        \item Base case is $n = 1$ (already checked).
        \item Assume proposition is true for $1 \leq n \leq m$ and consider the $n = m + 1$ case:
        \begin{align*}
          u_{m + 1}
          &= \sum_{k = 0}^{m + 1}p_ku_{m + 1 - k}\\
          &= \sum_{k = 1}^{m + 1}p_ku_{m + 1 - k}\quad \text{[since $p_0 = 0$]}\\
          &= p_{m + 1} + \sum_{k = 1}^mp_k\alpha\quad \text{[using induction hypothesis]}\\
          &= \alpha(1 - \alpha)^m + \sum_{k = 1}^m\alpha^2(1 - \alpha)^{k - 1}\\
          &= \alpha(1 - \alpha)^m + \alpha^2\left[\frac{1 - (1 - \alpha)^m}{1 - (1 - \alpha)}\right]\quad \text{[geometric series]}\\
          &= \alpha(1 - \alpha)^m + \alpha[1 - (1 - \alpha)^m]\\
          &= \alpha.
        \end{align*}
        So, if the proposition is true for $1 \leq n \leq m$, it is also true for $n = m + 1$.
        \item Hence, by induction, the proposition (5.2) is true for all integer $n \geq 1$. 
      \end{itemize}
      The geometric distribution can be thought of as the discrete analogue of the exponential distribution. The probability for renewal in a given time period is constant and $M(n) = \alpha n$ an for all integer $n$.
      \item Let
    \end{enumerate}

  \end{enumerate}
\end{document}