\documentclass[11pt,a4paper]{article}
\usepackage[margin=1in]{geometry}
\usepackage{amsfonts,amsmath,amssymb,suetterl}
\usepackage{lmodern}
\usepackage[T1]{fontenc}
\usepackage{fancyhdr}
\usepackage{float}
\usepackage[utf8]{inputenc}
\usepackage{physics}
\usepackage{fontawesome}
\DeclareUnicodeCharacter{2212}{-}
\usepackage{mathrsfs}
\usepackage[nodisplayskipstretch]{setspace}

\setstretch{1.5}
\setlength{\headheight}{15.2pt}

\pagestyle{fancy}
\fancyfoot[C]{\thepage}
\fancyfoot[L]{Course Organizer: R. J. HARRIS}
\fancyfoot[R]{rosemary.harris@qmul.ac.uk}

% \renewcommand{\footrulewidth}{0pt}
\renewcommand{\headrulewidth}{0pt}
\parindent 0ex
\setlength{\parskip}{1em}

\begin{document}
  \textbf{MTH734U/MTHM012} \hfill \textbf{Mathematical Sciences}\\
  \textbf{Semester B 2010 - 2011} \hfill \textbf{QMUL}
  \begin{center}
    \textbf{\huge Week 8}
  \end{center}
  \hrule \vspace{2mm} \hrule

  \section*{Key Objective:}
  \textit{Understand the arguments leading to forward and backward equations; be able to calculate the transition probability matrices corresponding to simple generators.}

  \section*{Background:}
  Carefully reread your lecture notes and supplement them with pages 394—397 of [T+K]. In particular, check the details of the calculations done in the lecture and make sure you can determine the probability transition matrix for an arbitrary $2\times 2$ generator. You should also think about how the birth-death processes considered earlier in Chapter 6 of [T+K] (and treated in the Probability Ill course) fit into the general theoretical framework.\par
  For a tricky challenge in addition to the problems overleaf, try to understand the question (not the answer!) in [T+K] Problem 6.4 - I think it's at best ambiguous.

  \newpage
  \section*{Problems:}
  \begin{enumerate}
    \item \textbf{[Part question from 2003 exam paper]}\\
    The transition probability matrix Of a time homogeneous, continuous time Markov chain $X(t)$ on state space $S$ is defined by
    $$
    (\vb{P}(t))_{i,j} = \text{Pr}\{X(s + t) = j|X(s) = i\}\quad \text{for $i,j\in S$}.
    $$
    Determine the transition probability matrix $\vb{P}(t)$ of the time homogeneous, continuous time Markov chain on state space $S = \{0, 1\}$ with generator
    $$
    \vb{G} =
    \begin{pmatrix}
      -1 & 1\\
      1 & -1
    \end{pmatrix}.
    $$
    \item \textbf{Follow-up}\\
    For the Markov chain of the previous question, find
    \begin{enumerate}
      \item $\text{Pr}\{X(t) = 1|X(0) = 0, X(3t) = 0\}$,
      \item $\text{Pr}\{X(t) = 1|X(0) = 0, X(3t) = 0, X(4t) = 0\}$.
    \end{enumerate}
    \item \textbf{Poisson process (again!)}\\
    Let $X(t)$ be a Poisson process with parameter $\lambda$. Write down the generator for this process and hence obtain the forward equations for $P_{0k}(t) = \text{Pr}\{X(t) = k\}$. Solve these equations, by the method of generating functions or otherwise, to show that $X(t)$ has (unsurprisingly) a Poisson distribution.
    \item \textbf{Laplace transform fun}\\
    Suppose the state space is finite. Consider the Laplace transform defined by
    $$
    \hat{P}_{ij}(\lambda) = \int_0^\infty e^{-\lambda t} P_{ij}(t)dt,\quad \lambda > 0.
    $$
    \begin{enumerate}
      \item Show
      $$
      \vb{\hat{P}}(\lambda) = (\lambda\vb{I} - \vb{A})^{-1},
      $$
      where $\vb{A}$ is the generator. Hints: Write $P_{ij}(t)$ in the Laplace transform as an integral, reverse the order of integration and use the backward equations.
      \item Use the result from (a) to find (again!) the transition probability matrix $\vb{P}(t)$ for the $2 \times 2$ generator considered in the lectures
      $$
      \vb{A} =
      \begin{pmatrix}
        -\alpha & \alpha\\
        \beta & -\beta
      \end{pmatrix}
      $$
      \item Starting from (a) justify the general solution
      $$
      \vb{P}(t) = \sum_{n = 0}^\infty \frac{\vb{A}^nt^n}{n!}.
      $$
    \end{enumerate}
    \textbf{(Hints available in tutorial on 9th March, solutions in tutorial on 16th March)}
  \end{enumerate}

  \newpage
  \textbf{MTH734U/MTHM012} \hfill \textbf{Mathematical Sciences}\\
  \textbf{Semester B 2010 - 2011} \hfill \textbf{QMUL}
  \begin{center}
    \textbf{\huge Solutions for Week 8}
  \end{center}
  \hrule \vspace{2mm} \hrule

  \section*{General Comments:}
  The first two questions should have been fairly straightforward this time. For the third question, do make sure that you're happy with the generator approach including solving the differential equation using the appropriate initial condition. The fourth question was a little harder but good for a challenge!

  \section*{Solutions to Problems:}
  \begin{enumerate}
    \item \textbf{[Part question from 2003 exam paper]}\\
    As discussed in the lecture, there are several possible approaches here; arguably the easiest is to use the forward equations
    $$
    \vb{P}^\prime(t) = \vb{P}(t)\vb{G}
    $$
    with
    $$
    \vb{P}(t)
    \begin{pmatrix}
      P_{00}(t) & P_{01}(t)\\
      P_{10}(t) & P_{11}
    \end{pmatrix},\quad
    \vb{G} =
    \begin{pmatrix}
      -1 & 1\\
      1 & -1
    \end{pmatrix}
    $$
    Using that
    \begin{align*}
      &P_{01}(t) = 1 - P_{00}(t)\\
      &P_{11}(t) = 1 - P_{10}(t),
    \end{align*}
    one gets the two differential equations
    \begin{align*}
      &P^\prime_{00}(t) = -2P_{00}(t) + 1\\
      &P^\prime_{10}(t) = -2P_{10}(t) + 1
    \end{align*}
    with initial conditions
    \begin{align*}
      &P_{00}(0) = 1\\
      &P_{10}(0) = 0.
    \end{align*}
    Solution of the differential equations yields
    \begin{align*}
      &P_{00}(t) = \frac{1}{2} + \frac{1}{2}e^{-2t}\\
      &P_{10}(t) = \frac{1}{2} - \frac{1}{2}e^{-2t},
    \end{align*}
    and hence
    $$
    \vb{P}(t) = \frac{1}{2}
    \begin{pmatrix}
      1 + e^{-2t} & 1- e^{-2t}\\
      1 - e^{-2t} & 1 + e^{-2t}
    \end{pmatrix}.
    $$
    \item \textbf{Follow-up}
    \begin{enumerate}
      \item The conditional probability can be expressed as
      \begin{align*}
        &\text{Pr}\{X(t) = 1|X(0) = 0, X(3t) = 0\}\\
        &\quad = \frac{\text{Pr}\{X(0) = 0, X(t) = 1, X(3t) = 0\}}{\text{Pr}\{X(0) = 0, X(3t) = 0\}}\\
        &\quad = \frac{\text{Pr}\{X(0) = 0\}\text{Pr}\{X(t) = 1|X(0) = 0\}\text{Pr}\{X(3t) = 0|X(t) = 1\}}{\text{Pr}\{X(0) = 0\}\text{Pr}\{X(3t) = 0 |X(0) = 0\}}\\
        &\quad = \frac{P_{01}(t)P_{10}(2t)}{P_{00}(3t)}\quad \text{[using time-gomogeneity]}\\
        &\quad = \frac{(1 - e^{-2t})(1 - e^{-4t})}{2(1 + e^{-6t})}.\quad \text{[using results of previous question]}
      \end{align*}
      \item A similar argument yields
      \begin{align*}
        \text{Pr}\{X(t) = 1|X(0) = 0, X(3t) = 0, X(4t) = 0\} 
        &= \frac{P_{01}(t)P_{10}(2t)P_{00}(t)}{P_{00}(3t)P_{00}(t)}\\
        &= \frac{(1 - e^{-2t})(1 - e^{-4t})}{2(1 + e^{-6t})}.
      \end{align*}
      The fact that the answer is the same as (a) should be obvious from the Markov property.\\
      -Why?
    \end{enumerate}
    \item \textbf{Poisson process (again!)}\\
    The generator is
    $$
    \begin{pmatrix}
      -\lambda & \lambda& 0 & 0 & \ldots\\
      0 & -\lambda & \lambda & 0 & \ldots\\
      0 & 0 & -\lambda & \lambda & \ldots\\
      \vdots & \vdots & \vdots & \vdots & \ddots
    \end{pmatrix}.
    $$
    and hence the forward equations are:
    \begin{equation}\tag{8.1}
      \begin{aligned}[b]
        P^\prime_{0, 0}(t) &= -\lambda P_{0,0}(t)\\
        P^\prime_{0, k}(t) &= -\lambda P_{0, k}(t) + \lambda P_{0, k-1}(t)\quad \text{for $k > 0$}
      \end{aligned}
    \end{equation}
    with initial conditions
    $$
    P_{0,0}(0) = 1,\quad\text{and}\quad P_{0,k}(0)\quad\text{for $k > 0$}.
    $$
    Note that (8.1) can be extended to include the $k = 0$ by imposing the "natural" boundary condition $P_{0, -1}(t) = 0$.\\
    One way to solve these equations is to define the generating functions
    \begin{equation}\tag{8.2}
      G(z, t) = \sum_{k = 0}^\infty P_{0,k}(t)z^k
    \end{equation}
    Then multiplying (8.1) by $z^k$ and summing over $k$ one finds
    \begin{align*}
      \sum_{k = 0}^\infty P^\prime_{0,k}(t)z^k
      &= \sum_{k = 0}^\infty z^k\{-\lambda P_{0,k}(t) + \lambda P_{0, k-1}(t)\}\\
      &= -\lambda\sum_{k = 0}^\infty P_{0,k}(t)z^k + \lambda z\sum_{k = 0}^\infty P_{0, k- 1}^tz^{k - 1}\\
      &= -\lambda\sum_{k = 0}^\infty P_{0,k}(t)z^k + \lambda z\sum_{j = 0}^\infty P_{0,j}(t)z^j
    \end{align*}
    which finally yields,
    \begin{equation}\tag{8.3}
      \frac{\partial G}{\partial t} = \lambda(z - 1)G,
    \end{equation}
    with initial condition is $G(z,0) = z^0 = 1$. The general solution to (8.3) is
    $$
    G(z, t) = C(z)e^{\lambda(z - 1)t}
    $$
    and the initial condition gives $C(z) = 1$ for all $z$. [Note that the normalization condition $\sum_{k = 0}^\infty P_{0,k}(t) = 1$ is only enough to fix $C(1) = 1$.] Hence we conclude that
    $$
    G(z, t) = e^{\lambda(z - 1)t} = e^{-\lambda t}\sum_{k = 0}\frac{(\lambda t)^k}{k!}z^k,
    $$
    which, by comparison with (8.2) implies that $X(t)$ has a Poission distribution:
    $$
    P_{0, k}(t) = e^{-\lambda t}\frac{(\lambda t)^k}{k!}.
    $$
    \item \textbf{Laplace Transform fun}
    \begin{enumerate}
      \item Defining the Kronecker delta as
      $$
      \delta_{ij} =
      \begin{cases}
        1 & \text{if $i = j$}\\
        0 & \text{if $i \neq j$ }
      \end{cases}
      $$
      the analysis proceeds as follows:
      \begin{align*}
        \hat{P}_{ij}
        &= \int_0^\infty e^{-\lambda t}P_{ij}(t)dt\\
        &= \int_0^\infty e^{-\lambda t}(P_{ij}(t) - P_{ij}(0))dt + \lambda^{-1}\delta_{ij}\\
        &= \int_0^\infty e^{-\lambda t}\left\{\int_0^t P^\prime_{ij}(s)ds\right\}dt + \lambda^{-1}\delta_{ij}\\
        &= \int_0^\infty P^\prime_{ij}(s)\left\{\int_s^\infty e^{-\lambda t}dt\right\}ds + \lambda^{-1}\delta_{ij}\quad \text{[changing order of integration]}\\
        &= \int_0^\infty P^\prime_{ij}(s)\lambda^{-1}e^{-\lambda s}ds + \lambda^{-1}\delta_{ij}\\
        &= \sum_{k}\int_0^\infty A_{ik}P_{kj}(s)\lambda^{-1}e^{-\lambda s}ds + \lambda^{-1}\delta_{ij}\quad\text{[using backward equation]}\\
        &= \lambda^{-1}\sum_k A_{ik}\hat{P}_{kj}(s) + \lambda^{-1}\delta_{ij}.
      \end{align*}
      In matrix notation this is just
      $$
      \lambda\hat{\vb{P}}(\lambda) = \vb{A}\hat{\vb{P}}(\lambda) + \vb{I}
      $$
      which is straightforwardly re-arranged to yield
      $$
      \hat{\vb{P}}(\lambda) = (\lambda\vb{I} - \vb{A})^{-1}.
      $$
      \item For the generator
      $$
      \vb{A} =
      \begin{pmatrix}
        -\alpha & \alpha\\
        \beta & -\beta
      \end{pmatrix}
      $$
      we have
      $$
      \lambda\vb{I} - \vb{A} =
      \begin{pmatrix}
        \lambda + \alpha & -\alpha\\
        -\beta & \lambda + \beta
      \end{pmatrix}
      $$
      and inverting this gives
      \begin{align*}
        (\lambda\vb{I} - \vb{A})^{-1} 
        &= \frac{1}{\det(\lambda\vb{I} - \vb{A})}
        \begin{pmatrix}
          \lambda + \beta & \alpha\\
          \beta & \lambda + \alpha
        \end{pmatrix}\\
        &= \frac{1}{\lambda(\lambda + \alpha + \beta)}
        \begin{pmatrix}
          \lambda + \beta & \alpha\\
          \beta & \lambda + \alpha
        \end{pmatrix}.
      \end{align*}
      As usual, the tricky part is doing the inverse Laplace transform [once again, equation (1.1) in the Week 1 solutions may be useful]. We start by using partial fractions to write
      $$
      \frac{1}{\lambda(\lambda + \alpha + \beta)} = \frac{1}{\alpha + \beta}\left\{\frac{1}{\lambda} - \frac{1}{\lambda + \alpha + \beta}\right\}.
      $$
      Then we have
      \begin{align*}
        (\lambda\vb{I}- \vb{A})^{-1}
        &= \frac{1}{\alpha + \beta}
        \begin{pmatrix}
          \frac{\lambda + \beta}{\lambda} - \frac{\lambda + \beta}{\lambda + \alpha + \beta} & \frac{\alpha}{\lambda} - \frac{\alpha}{\lambda + \alpha + \beta}\\
          \frac{\beta}{\lambda} - \frac{\beta}{\lambda + \alpha + \beta} & \frac{\lambda + \alpha}{\lambda} - \frac{\lambda + \alpha}{\lambda + \alpha + \beta}
        \end{pmatrix}\\
        &= \frac{1}{\alpha + \beta}
        \begin{pmatrix}
          1 + \frac{\beta}{\lambda} - \frac{\lambda + \beta}{\lambda + \alpha + \beta} & \alpha\left(\frac{1}{\lambda} - \frac{1}{\lambda + \alpha + \beta}\right)\\
          \beta\left(\frac{1}{\lambda} - \frac{1}{\lambda + \alpha + \beta}\right) & 1 + \frac{\alpha}{\lambda} - \frac{\lambda + \alpha}{\lambda + \alpha + \beta}
        \end{pmatrix}\\
        &= \frac{1}{\alpha + \beta}
        \begin{pmatrix}
          \frac{\beta}{\lambda} + \frac{\alpha}{\lambda + \alpha + \beta} & \alpha\left(\frac{1}{\lambda} - \frac{1}{\lambda + \alpha + \beta}\right)\\
          \beta\left(\frac{1}{\lambda} - \frac{1}{\lambda + \alpha + \beta}\right) & \frac{\alpha}{\lambda} + \frac{\beta}{\lambda + \alpha + \beta}
        \end{pmatrix}\\
        &= \frac{1}{\alpha + \beta}
        \begin{pmatrix}
          \int_0^\infty(\beta e^{-\lambda t} + \alpha e^{-(\lambda + \alpha + \beta)})dt & \int_0^\infty(\alpha e^{-\lambda t} - \alpha e^{-(\lambda + \alpha + \beta)})dt\\
          \int_0^\infty(\beta e^{-\lambda t} - \beta e^{-(\lambda + \alpha + \beta)t})dt & \int_0^\infty(\alpha e^{-\lambda t} + \beta e^{-(\lambda + \alpha + \beta)t}dt)
        \end{pmatrix}\\
        &= \frac{1}{\alpha + \beta}\int_0^\infty e^{-\lambda t}
        \begin{pmatrix}
          \beta + \alpha e^{-(\alpha + \beta)t} & \alpha - \alpha e^{-(\alpha + \beta)t}\\
          \beta - \beta e^{-(\alpha + \beta)t} & \alpha + \beta e^{-(\alpha + \beta)t}
        \end{pmatrix}dt,
      \end{align*}
      where we interpret the integral of a matrix as the matrix of integrals of the components of the matrix. From here, we can read off
      $$
      \vb{P}(t) = \frac{1}{\alpha + \beta}
      \begin{pmatrix}
        \beta + \alpha e^{-(\alpha + \beta)t} & \alpha - \alpha e^{-(\alpha + \beta)t}\\
        \beta - \beta e^{-(\alpha + \beta)t} & \alpha + \beta e^{-(\alpha + \beta)t}
      \end{pmatrix}
      $$
      in agreement with the other methods.
      \item We have
      \begin{align*}
        \hat{\vb{P}}(t)
        &= (\lambda\vb{I} - \vb{A})^{-1}\\
        &= \lambda^{-1}(\vb{I} - \lambda^{-1}\vb{A})^{-1}\\
        &= \lambda^{-1}\sum_{n = 0}^\infty(\lambda^{-1}\vb{A})^n\\
        &= \lambda^{-1}\sum_{n = 0}^\infty\vb{A}^n\lambda^{-n}\\
        &= \sum_{n = 0}^\infty\frac{A^n}{n!}\int_0^\infty e^{-\lambda t}t^ndt\\
        &= \int_0^\infty e^{-\lambda t}\sum_{n = 0}^\infty\frac{\vb{A}^nt^n}{n!}dt,
      \end{align*}
      and it follows that
      $$
      \vb{P}(t) = \sum_{n = 0}^\infty \frac{\vb{A}^nt^n}{n!}.
      $$
    \end{enumerate}
  \end{enumerate}
\end{document}